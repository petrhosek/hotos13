\section{Introduction}
\label{sec:intro}

\begin{structure}
  \item Full virtualization and para-virtualization are too heavyweight
  \item SFI sandbox as lightweight replacement providing isolation,
    security and portability
  \item Missing process abstraction and the solution
\end{structure}

%Virtualization has become de facto standard for provisioning
%infrastructure in datacenters~\cite{barosso:datacenter}.

With the rising popularity of cloud computing, the prevalence of
virtualization is steadily growing. While \emph{Infrastructure as a
Service} (IaaS) model assumes provisioning an entire operating system
using system virtual machines (\eg Xen or KVM), \emph{Platform as a
Service} (PaaS) models typically relies an execution environment in the
form of application virtual machine (\eg JVM or Python). The advantage
of the former is finer degree of control over the entire execution stack
while the advantage of the latter is a better resource utilization as we
can run large number of application--potentially from different
customers---on a single machine. The existing PaaS platforms such as
Google App Engine already uses multi-tenancy where several applications
share the same physical or virtual machine.  However, to ensure security
and isolation, the number of supported platforms is limited to Java,
Python and Go using custom sandboxed runtimes.

In this position paper, we propose an alternative which combines
advantages of both approaches. We argue for an application virtual
machine capable of executing native code providing the standard POSIX
interface. We aim to combine \emph{software fault isolation} (SFI) for
security and isolation properties with microkernel service oriented
architecture to provide process management, file and network access.
Based on top of Native Client~\cite{yee:ieee-sp09}, state of the art
industrial strength SFI implementation, this solution can provide
security and performance assurances comparable to existing system
virtual machines at fractional \emph{trusted code base} (TCB) size which
allows for easier verification and maintenance. 

Such execution environment could form a basis for a new kind of
POSIX-compatible PaaS native platform, which will allow for execution of
high-performance applications. Such platform will be more efficient than
traditional virtualization and some language based sandboxes while more
flexible than Linux-based systems, akin to virtual machines.
Furthermore, native code support will enable use of existing C/C++
libraries and legacy code.  Example of such service is Google
Exacycle~\cite{exacycle:google}, which provides PaaS platform for
large-scale, embarrasingly parallel batch computations.

%The solution based on Native Client would allow executing native
%POSIX-compliant code at near native speed while providing the same
%security and isolation guarantees. This will allow the use of wide
%range of existing legacy software without the overhead of running (and
%managing) the entire VM.

%i like NaCl as a posix-like platform-as-a-service a little better,
%though restricting to the posix-y subset sort of works wrt running on a
%real posix-y OS too, i guess.

%more efficient than virtualization and some language based sandboxes.
%more flexible than LBS, akin to VMs
%not just novelty, but TCB minimization

The rest of this paper is organized as follows.
Section~\ref{sec:overview} gives an overview of our approach, \ldots.
Then, Section~\ref{sec:prototype} presents \ldots. Finally,
Section~\ref{sec:related} discusses related work and
Section~\ref{sec:conclusion} concludes.

%Furthermore, our approach gives greater level control which is so
%important in cloud deployments.

%In recent years, we have witnessed a boom of virtualization. Most of the
%server infrastructure which is now being used runs virtualized.
%However, virtualization still failed to attract desktop market despite
%obvious advantages such as portability, security, reliability and
%isolation. We believe this is because the existing virtualization
%solutions are too heavyweight and cumbersome. They require the user run
%an entire operating system which is suitable for infrastructure
%provisioning in data centers as well as certain cloud applications, but
%is simply too cumbersome for desktop and client applications.

%In this paper, we propose a more lightweight alternative. Instead of
%running the full operating system, our goal is to combine aspects of
%microkernel and monolithic kernel design with software fault isolation
%techniques to provide lightweight solution for virtualization of
%existing applications in the user space. Furthemore, our proposed
%approach allows applications to be run on top of three major operating
%system without recompilation.
