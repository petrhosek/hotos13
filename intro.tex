\section{Introduction}
\label{sec:intro}

\begin{structure}
  \item Full virtualization and para-virtualization are too heavyweight
  \item SFI sandbox as lightweight replacement providing isolation,
    security and portability
  \item Missing process abstraction and the solution
\end{structure}

In recent years, we have witnessed a boom of virtualization. Most of the
servier infrastructure which is now being used is running virtualized.
However, virtualization still failed to attract desktop market despite
obvious advantages such as portability, security, reliability and
isolation. We believe this is because the existing virtualization
solutions are too heavyweight and cumbersome. They allow to run entire
operating system which is suitable for infrastructure provisioning in
data center and cloud applications, but is simply too cumbersome for
desktop and client applications.

In this paper, we propose a more lightweight alternative. Instead of
running full operating system kernel, our goal is to combine aspects of
microkernel and monolithic kernel design with software fault isolation
techniques to provide lightweight solution for virtualization of
existing applications in the user space.  Furthemore, our proposed
approach allows applications to be run on top of three major operating
system without recompilation.

The rest of this paper is organized as follows.
Section~\ref{sec:overview} gives an overview of our approach, \ldots.
Then, Section~\ref{sec:prototype} presents \ldots. Finally,
Section~\ref{sec:related} discusses related work and
Section~\ref{sec:conclusion} concludes.
