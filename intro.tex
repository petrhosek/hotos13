\section{Introduction}
\label{sec:intro}

%With the rising popularity of cloud computing, the popularity of
%virtualization is steadily growing as a way to implement
%\emph{multi-tenancy}.

%In recent years, virtualization has become crucialy important for \ldots
%multi-tenancy, cloud computing.

Virtualization, in its many forms, has become the universal solution to
a number of problems including security and portability.  With the
rising popularity of cloud computing, virtualization has also become
crucially important as a way to ensure \emph{multi-tenant security}. The
XaaS applications can run either on top of virtual machine monitors,
where the customer is provided with the entire operating system (\eg
Google Compute Engine) or on top of existing language runtime (\eg
Google App Engine).

While the operating system-level virtualization is desirable in cases
when we require control over the entire execution stack, it can be
simply too cumbersome when we need to execute only a single
application.  A typical OS installation may require storage space in
the order of GB plus additional CPU and memory overhead for running
the kernel and system services. Furthermore, the guest OS may require
frequent maintenance (\eg installing updates, managing the
configuration).  Language runtimes on the other hand, while being more
lightweight and easier to use, typically only target a single
high-level language or language family and similarly require continual
security maintenance \cite{java0day2013}.

In this position paper, we argue for an application virtual machine (VM)
capable of running existing POSIX-based applications. The main goal of
our approach is to offer the same security and isolation assurances as
hardware-based hypervisors (\eg Xen), but implemented entirely in the
user space on top of existing monolithic kernel resembling language
runtimes (\eg JVM). Furthermore, by providing the clean embedding
interface, such a VM could be integrated into both new and existing
systems to allow safe native code execution.

%With the rising popularity of cloud computing, virtualization has become
%crucially important as a way to provide \emph{multi-tenancy}. While the
%\emph{Infrastructure as a Service} (IaaS) model assumes provisioning an
%entire operating system using virtual machine monitors (\eg Google
%Compute Engine), the \emph{Platform as a Service} (PaaS) model typically
%relies on a language runtimes (\eg Google App Engine). The~advantage of
%the former is a finer degree of control over the entire application
%stack while the latter allows for better resource utilization. A~modern
%hypervisor uses hardware virtualization technology to enforce isolation
%and resource allocation; in case of language runtimes, ad-hoc sandboxing
%mechanisms are typically being used to provide the same guarantees,
%which results in a limited number of supported platforms (\eg Java,
%Python and Go in case of Google App Engine).

%As cloud providers can use multi-tenancy run large number of
%applications---potentially coming from different customers---under a
%single operating system instance. The existing PaaS platforms such as
%Google App Engine already uses multi-tenancy where several applications
%share the same physical or virtual machine.  However, to ensure the
%security and isolation, the number of supported platforms is limited to
%Java, Python and Go using custom sandboxed runtimes.

%The existing PaaS platforms such as Google App Engine already uses
%multi-tenancy where several applications share the same physical or
%virtual machine.  However, to ensure the security and isolation, the
%number of supported platforms is limited to Java, Python and Go using
%custom sandboxed runtimes.

Such an execution environment could form the basis for a new kind of
POSIX compliant PaaS native platform which would allow for execution of
high-performance applications. Such a platform would be more efficient
than traditional hardware virtualization platforms and some language
based sandboxes while being more flexible than Linux-based systems, and
would be akin to virtual machines.  Furthermore, native code support
enables the use of existing C/C++ libraries and legacy code as well as
extended instruction sets (\eg SSE or AVX) for high-performance
applications. An example of such a service is Google
Exacycle,\footnote{See
\url{http://research.google.com/university/exacycle_program.html} and
\url{http://googleresearch.blogspot.com/2011/04/1-billion-core-hours-of-computational.html}.}
which provides a PaaS platform for large-scale, embarrassingly parallel
batch computations.

%Furthermore, native code support together with full POSIX interface
%allows running existing runtimes on top of our execution environment
%without the necessity of an ad-hoc sandboxing mechanism since the
%environment itself already provides sandboxing capabilities. This has a
%number of advantages.  First, it is possible to use ``vanilla'' language
%runtimes which allows for upgrading to a newer version immediately after
%they are released significantly reducing the risk of zero-day
%vulnerabilities. Second, it allows provisioning of wider range of
%runtimes potentially attracting more customers, even those using less
%popular languages and platforms.  Third, it will allow use of native
%extensions supported by most language runtimes, but not always available in
%PaaS environments for security reasons.

%The solution based on Native Client would allow executing native
%POSIX-compliant code at near native speed while providing the same
%security and isolation guarantees. This will allow the use of wide
%range of existing legacy software without the overhead of running (and
%managing) the entire VM.

%i like NaCl as a posix-like platform-as-a-service a little better,
%though restricting to the posix-y subset sort of works wrt running on a
%real posix-y OS too, i guess.

%more efficient than virtualization and some language based sandboxes.
%more flexible than LBS, akin to VMs
%not just novelty, but TCB minimization

The rest of this paper is organized as follows.
Section~\ref{sec:motivation} sets out the motivation for our proposed
solution while Section~\ref{sec:overview} gives an overview of the
technical approach.  Then, Section~\ref{sec:prototype} presents a
possible prototype implementation. Finally, Section~\ref{sec:related}
discusses some of the related work and Section~\ref{sec:conclusion}
concludes.

%Furthermore, our approach gives greater level control which is so
%important in cloud deployments.

%In recent years, we have witnessed a boom of virtualization. Most of the
%server infrastructure which is now being used runs virtualized.
%However, virtualization still failed to attract desktop market despite
%obvious advantages such as portability, security, reliability and
%isolation. We believe this is because the existing virtualization
%solutions are too heavyweight and cumbersome. They require the user run
%an entire operating system which is suitable for infrastructure
%provisioning in data centers as well as certain cloud applications, but
%is simply too cumbersome for desktop and client applications.

%In this paper, we propose a more lightweight alternative. Instead of
%running the full operating system, our goal is to combine aspects of
%microkernel and monolithic kernel design with software fault isolation
%techniques to provide lightweight solution for virtualization of
%existing applications in the user space. Furthemore, our proposed
%approach allows applications to be run on top of three major operating
%system without recompilation.
