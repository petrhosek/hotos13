\section{Overview}
\label{sec:overview}

\begin{structure}
\item High-level overview of our approach.
\item What is NaCl and how does it work?
\item NaCl architecture and components (with figure)
\end{structure}

%we need to talk about files as a service -- naturally comes up in the
%microkernel structure stuff -- and how file service handles namespaces
%but file I/O is essentially native

We aim to provide an application virtual machine (VM) capable of running
existing POSIX based applications with minimal or no changes at all. The
main goal of our approach is to provide the same security and isolation
assurances as hardware-based hypervisors (\eg Xen or KVM), but
implemented entirely in the user space on top of existing monolithic
kernel resembling language runtimes (\eg JVM).

\todo{Talk about files as a service---naturally comes up in the
microkernel structure stuff---and how file service handles namespaces
but file I/O is essentially native.}

We argue that the optimal way of implementing such VM is to combine SFI
with microkernel style design.
\todo{Try to come up with a better argumentation.}
Compared to on-the-metal microkernels, there are number of significant
differences. First of all, we rely on existing monolithic kernel (\ie
Linux, OS X or Windows) to do the ``heavy lifting'', in particular
handling hardware eliminating the need for device drivers. Second,
various microkernel-related overheads (\eg context switch) still exist
which may introduce additional performance overhead. The benefits of our
approach are structural; since the VM is implemented entirely in the
user space, it doesn't require any special privileges which makes
component/module replacement significantly easier. The reduced TCB also
size allows for easier security audits.

\begin{figure}
\centering
%\includegraphics{architecture}
\caption{Overview of core NaCl components}
\label{fig:architecture}
\end{figure}

Our proposed solution will be based on top of Native Client (commonly
abbreviated ``NaCl'' when used as an adjective), an open-source
production-quality SFI-based sandbox for the safe execution of
untrusted, multi-threaded user-level machine code. While primarily aimed
towards executing native compiled code inside the web browser, the NaCl
core components, depicted in Figure~\ref{fig:architecture}, can be also
used as a general-purpose sandbox.

\todo{Do mention that NaCl has been security audited, this was possible
because of the small TCB size.}

Native Client combines two sandboxing mechanisms. The inner sandbox
supports restricted subset of x86, x86-64 and ARM ISA defined by a set
of constraints, a modified compilation tool chain that observes these
constrains, and a static validator that verifies that the constrains
have been followed. To sandbox data accesses, Native Client uses
different hardware mechanisms depending on target hardware
platform~\cite{yee:ieee-sp09,sehr:usenix-sec10}. For additional
security, Native Client optionally uses additional outer sandbox
implemented as a traditional OS system-call interception (\eg using
seccomp~\cite{seccomp:linux} or
seccomp\mbox{-}bpf~\cite{seccomp-bpf:linux}).

The sandbox is implemented as a part of \emph{service runtime}, a native
executable which provides resource abstractions to isolate NaCl
applications from host resources and underlying operating system
interfaces. Even though it shares a process with the contained NaCl
module, the sandbox prevents untrusted code from inappropriate accesses.
The service runtime functionality is only accessible through system
calls which resemble standard kernel system calls by performing a
``context switch'' from untrusted to trusted code, even though they use
a different hardware mechanism. The service runtime already follows
microkernel design by implementing many aspects of the sandbox in terms
of services. All services are implemented using Simple RPC (SRPC), a
simple and basic remote method invocation mechanism supporting only
basic data types (\ie \lstinline`int`, \lstinline`float`,
\lstinline`char` in addition to arrays and internal descriptors).

\todo{Talk about portability; the same NaCl application can run on
Linux, OS X and Windows as long as the hardware architecture is the
same, with PNaCl even this restriction is lifted (compare to JVM and
other language runtimes).}

%SRPC is implemented entirely in untrusted code on top the
%IMC (Inter-Module Communications), Native Client's reliable datagram
%service. External data representation strategies can easily be layered
%on top of SRPC.

To abstract away the internal details Native Client provides
\emph{Integrated Runtime} (IRT), an untrusted library providing a
stable, backward compatible interface to a NaCl module. IRT ensures that
once compiled, NaCl module will run forever even with a newer version of
service runtime. Facilities provided to untrusted code may be
implemented either as NaCl system calls or as SRPC services. IRT
abstracts these differences and provides unified interface. The
advantage of using services over system calls is extensibility and
easier development, the disadvantage is primarily performance overhead
as each service method invocation requires several system calls (\ie
sequence of \lstinline`sendmsg`/\lstinline`recvmsg` invocations).

The Application Binary Interface (ABI) exposed to the untrusted code is
modeled after POSIX ABI and includes subset of the POSIX thread
interface as well as common POSIX file I/O interfaces.  However,
application programmers do not typically use the ABI directly, instead
they use one of the provided C libraries (\ie Newlib or Glibc) which hide
away internal details and provide common interface.

%The NaCl architecture resembles existing micro and hybrid kernels.
%However, compared to on-the-metal microkernels, there are number of
%significant differences. First of all, Native Client sits on top of an
%existing monolithic kernel (\ie Linux, OS X or Windows), so the various
%microkernel-related overheads (\eg context switch) still exist.

%The solution based on Native Client would allow executing native
%POSIX-compliant code at near native speed while providing the same
%security and isolation guarantees. This will allow the use of wide
%range of existing legacy software without the overhead of running (and
%managing) the entire VM.

%Based on top of Native Client~\cite{yee:ieee-sp09}, state of the art
%industrial strength SFI implementation, this solution can provide
%security and performance assurances comparable to existing system
%virtual machines at a fractional \emph{trusted code base} (TCB) size
%which allows for easier verification and maintenance. 

%The NaCl architecture, depicted in Figure~\ref{fig:architecture},
%comprises of many different components resembling different components
%of an operating system. The component which fulfills the role of kernel
%is the \emph{service runtime}.
% largely resembles existing monolithic and micro-kernel design.

%The NaCl architecture resembles existing micro and hybrid kernels.
%However, compared to on-the-metal microkernels, there are number of
%significant differences. First of all, Native Client sits on top of an
%existing monolithic kernel (\ie Linux, OS X or Windows), so the various
%microkernel-related overheads (\eg context switch) still exist.

%\todo{Explain how the NaCl architecture compares with on-the-metal
%  micro-kernel designs, viz: Native Client is on top of an existing
%  monolithic kernel, so the various microkernel-related overheads
%  (context switch, etc) still exist.  The benefits are structural:
%  easier replacement of components/modules, easier security audits.}

%Native Client follows micro-kernel design by implementing many aspects
%of the sandbox in terms of services. All services are implemented using
%SRPC (Simple RPC) which \ldots implemented on top of the IMC
%(Inter-Module Communications). SRPC abstraction is implemented entirely
%in untrusted code. SRPC provides a convenient syntax for declaring
%procedural interfaces between NaCl modules, supporting a few basic types
%(int, float, char) as well as arrays in addition to NaCl descriptors.
%More complex types and pointers are not supported. External data
%representation strategies can easily be layered on top of SRPC.

%\subsection{Service Runtime}

%The service runtime is a native executable which provides resource
%abstractions to isolate NaCl applications from host resources and
%underlying operating system interfaces. Even though it shares a process
%with the contained NaCl module, the service runtime prevents untrusted
%code from inappropriate accesses using a combination of various
%techniques. % which differ for various target architectures. The service
%% runtime trusted code and data are only accessible through a controlled
%% interface.

%The service runtime functionality is only accessible through system
%calls which resemble standard kernel system calls by performing a contex
%switch from untrusted to trusted code, even though they use different
%hardware mechanism~\cite{yee:ieee-sp09,sehr:usenix-sec10}. The service
%runtime Application Binary Interface (ABI) exposed to the untrusted code
%is modeled after POSIX ABI and includes subset of the POSIX thread
%interface as well as common POSIX file I/O interfaces. This makes it
%easy to port existing POSIX application over to Native Client with
%minimal or no changes at all.

%%While Native Client already resembles existing hybrid kernel design
%%providing facilities such as multi-threading and
%%virtual memory management, it lacks two important features, process
%%management and shared memory. \todo{Not sure whether shared memory
%%really that important?}

%\todo{Each service runtime instance has two channels---secure command
%channel used by reverse service to control the execution, in particular
%to load the NaCl executable and intergrated runtime, start and stop the
%module execution; and reverse channel used by service runtime to access
%kernel functionality.}

%\subsection{Reverse Service}

%% While service runtime is the in-process sandbox enforcing fault
%% isolation and providing facilities such basic IPC (SRPC),
%% virtual memory, resource abstraction or the system call interface, it is
%% the \emph{reverse service} which provides inter-process services such as
%% file access, application IPC or process support.

%The \emph{reverse service} represents the embedding interface, providing
%core facilities such as process management, file system access or
%application. IPC % thereby resembling the traditional micro-kernels. 
%The name comes from the fact that unlike other service which act as a
%server, the reverse service starts as a client connecting to a service
%runtime which acts a server and only then is the connection
%\emph{reversed} and reverse service becomes the server. This setup is
%necessary since service runtime is not allowed to establish outside
%connection as one of the security measures.

%Each service runtime instance has a connection to reverse service. The
%\emph{initial} service runtime has a connection to the trusted reverse
%service \ldots while child processes might be using a different instance
%of the reverse service. This allows \ldots injecting a proxy which
%intercepts some (or all) calls to the actual reverse service or provide
%entirely different reverse service implementation running in untrusted
%code.

%\subsection{Integrated Runtime}

%Integrated Runtime (IRT) is an untrusted library providing a stable,
%backward compatible interface to NaCl module. IRT ensures that once
%compiled, NaCl module will run forever even with a newer version of NaCl
%runtime. Application programmers do not typically use IRT directly,
%instead they use one of the provided C libraries (Newlib or Glibc) which
%hide away internal details and provide common interface.

%Facilities provided to untrusted code may be implemented either as NaCl
%systeem calls or as SRPC services. IRT abstracts these differences and
%provides unified interface. The advantage of using services over system
%calls is extensibility and easier development, the disadvantage is
%primarily performance overhead as each service method invocation
%requires several system calls (\ie sequence of \lstinline`sendmsg`
%/\lstinline`recvmsg` invocations).

%\todo{Describe the role of IRT. Describe name and kernel services, their
%role.}
