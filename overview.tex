\section{Native Client as a VM}
\label{sec:overview}

Native Client comprises of a sandbox which supports supports restricted
subset of x86, x86-64 and ARM ISA defined by a set of constraints, a
modified compilation tool chain generating code that observes these
constrains, and a static validator that verifies that the constrains
have been followed.  The sandbox is implemented as the part of the
\emph{service runtime}, a native executable which provides resource
abstractions to isolate NaCl applications from host resources and
underlying operating system interfaces. Even though it shares a process
with the contained NaCl module, the service runtime prevents untrusted
code from inappropriate accesses by sandboxing data accesses using
various hardware mechanisms. The service runtime functionality is
accessible through system calls which resemble standard kernel system
calls by performing a ``context switch'' from untrusted to trusted code,
even though they use a different hardware mechanism.

%The Application Binary Interface (ABI) exposed to the untrusted code is
%modeled after POSIX ABI and includes subset of the POSIX thread
%interface as well as common POSIX file I/O interfaces.

\begin{figure}
\centering
%\includegraphics{architecture}
\caption{Overview of core NaCl components}
\label{fig:overview}
\end{figure}

The service runtime could already be seen as a microkernel. The three key
abstractions that a microkernel should provide according to
Liedtke~\cite{liedtke:sosp93} are address spaces, threads and
inter-process communication (IPC). The service runtime manages the
process' address space to ensure that all data accesses are properly
sandboxed.  Threads are available through a subset of POSIX thread
interface.  Finally, it implements certain services using Simple RPC
(SRPC), a simple and basic remote method invocation mechanism, which
could be used as an IPC mechanism. \todo{Mention that internal
descriptors could be seen as an efficient IPC mechanism implementation.}

What is missing to allow Native Client being used as a portable VM for
native applications are standard POSIX interfaces; the existing
Application Binary Interface (ABI) exposed to the untrusted code
contains only a subset of POSIX ABI including the POSIX thread interface
and common POSIX file I/O interfaces. While this allows porting some of
the existing POSIX based code, it is not enough to enable use of
existing applications simply by recompiling them using NaCl tool chain.

Rather than implementing the missing interfaces as a part of the service
runtime, unnecessarily increasing the TCB size, we believe that it is
possible to implement the missing POSIX interfaces using the microkernel
design by implementing the majority of functionality in terms of
services (or \emph{servers} in microkernel terms).  These services
should be a part of \emph{embedding interface}, a set of services the
embedder of Native Client provides to the NaCl runtime. The service
runtime may then mediate access to (some of) these services. The
embedders may also omit providing some of these services in situation
when certain interfaces are not available (\eg filesystem or network
access)

To abstract away the internal details Native Client provides
\emph{Integrated Runtime} (IRT), an untrusted library providing a
stable, backward compatible interface to a NaCl module. IRT ensures that
once compiled, NaCl module will run forever even with a newer version of
service runtime. Facilities provided to untrusted code may be
implemented either as NaCl system calls or as SRPC services. IRT
abstracts these differences and provides single unified interface.
Therefore, IRT could be seen as a part of embedding interface. We expect
embedders to provide their own version of IRT which will use the
services provided as a part of embedding interface. It is also possible
to implement certain POSIX interfaces entirely as a part of IRT (\eg
in-memory filesystem).

%The features we want to provide are flexibility and maintainability.
%\ldots embedding interface \ldots allows to use \ldots in various setups
%\ie executing native code within a web page, lightweight virtualization
%in cloud provisioning

%Different implementations of embedding interface, one for browser, one
%for server, one for mobile phone, etc.

%What is missing to provide untrusted code with the POSIX compliant
%interface are services (or \emph{servers} in microkernel terms) for
%process support, file system and network access. These services should
%be a part of \emph{embedding interface}, a set of services the embedder
%of Native Client provides to the NaCl runtime. The service runtime may
%then mediate access to (some of) these services.

%Although Native Client already provides a subset of POSIX interface,
%which allows porting some of the existing POSIX based code, it is still
%missing certain interfaces which makes it difficult to run existing
%applications simply by recompiling them using NaCl tool chain.  What is
%missing to provide untrusted code with the POSIX compliant interface are
%services (or \emph{servers} in microkernel terms) for process support,
%file system and network access. These services should be a part of
%\emph{embedding interface}, a set of services the embedder of Native
%Client provides to the NaCl runtime. The service runtime may then
%mediate access to (some of) these services.

%The Application Binary Interface (ABI) exposed to the untrusted code is
%modeled after POSIX ABI and includes subset of the POSIX thread
%interface as well as common POSIX file I/O interfaces.  However,
%application programmers do not typically use the ABI directly, instead
%they use one of the provided C libraries (\ie Newlib or Glibc) which hide
%away internal details and provide common interface.

%To abstract away the internal details Native Client provides
%\emph{Integrated Runtime} (IRT), an untrusted library providing a
%stable, backward compatible interface to a NaCl module. IRT ensures that
%once compiled, NaCl module will run forever even with a newer version of
%service runtime. Facilities provided to untrusted code may be
%implemented either as NaCl system calls or as SRPC services. IRT
%abstracts these differences and provides unified interface. The
%advantage of using services over system calls is extensibility and
%easier development, the disadvantage is primarily performance overhead
%as each service method invocation requires several system calls (\ie
%sequence of \lstinline`sendmsg`/\lstinline`recvmsg` invocations).

%Native Client combines two sandboxing mechanisms. The inner sandbox
%supports restricted subset of x86, x86-64 and ARM ISA defined by a set
%of constraints, a modified compilation tool chain generating code that
%observes these constrains, and a static validator that verifies that
%the constrains have been followed. To sandbox data accesses, Native
%Client uses different hardware mechanisms depending on target hardware
%platform~\cite{yee:ieee-sp09,sehr:usenix-sec10}. For additional
%security, Native Client optionally uses additional outer sandbox
%implemented as a traditional OS system-call interception (\eg using
%seccomp~\cite{seccomp:linux} or
%seccomp\mbox{-}bpf~\cite{seccomp-bpf:linux}).

%The sandbox is implemented as the part of \emph{service runtime}, a
%native executable which provides resource abstractions to isolate NaCl
%applications from host resources and underlying operating system
%interfaces. Even though it shares a process with the contained NaCl
%module, the sandbox prevents untrusted code from inappropriate accesses.
%The service runtime functionality is only accessible through system
%calls which resemble standard kernel system calls by performing a
%``context switch'' from untrusted to trusted code, even though they use
%a different hardware mechanism. The service runtime implements some
%aspects of the sandbox in terms of services (\eg name service). All
%services are implemented using Simple RPC (SRPC), a simple and basic
%remote method invocation mechanism supporting only basic data types (\ie
%\lstinline`int`, \lstinline`float`, \lstinline`char` in addition to
%arrays and internal descriptors).

%In this position paper, we argue for a lightweight VM capable of
%executing native code providing the standard POSIX interface.  We aim to
%combine \emph{software fault isolation} (SFI)~\cite{wahbe:sosp93} for
%security and isolation properties with microkernel-style service
%oriented architecture to provide process management, file and network
%access. Based on top of Native Client~\cite{yee:ieee-sp09}, a state of
%the art industrial strength SFI implementation, this solution can
%provide security and performance assurances comparable to existing
%system virtual machines at a fractional \emph{trusted code base} (TCB)
%size which allows for easier verification and maintenance.

%Hypervisors these days rely on virtualization hardware to deliver
%reasonable performance overhead. This makes the hypervisor the most
%privileged process effectively replacing the OS~\cite{heiser:hotos11}.
%Needles to say, it makes the hypervisor a critical part of the entire
%execution stack.  It does not matter whether the hypervisor is over 1
%MLOC~\cite{barham:sosp03} or just a few kLOC~\cite{steinberg:eurosys10},
%any vulnerability in the hypervisor implementation may potentially
%subvert the security of all hosted operating systems. This is a serious
%issue in multi-tenant environment where different host operating systems
%may run software from different customers.

%We aim for a different approach. Instead of virtualization hardware, we
%plan to employ software fault isolation~\cite{wahbe:sosp93} to direct
%all system interactions through controlled interfaces. This allows
%implementing the VM entirely in the userspace as a regular process
%without any access to privileged CPU features. Even if the security of
%the VM is compromised, the attacker is still limited by the userspace
%boundary. To further increase security, it is possible to employ other
%sandboxing mechanisms such as seccomp-bpf~\cite{seccomp-bpf:linux}.

%We believe it is indeed possible to implement a simple and secure
%virtualization environment sharing the approach with microkernel-based
%systems. These systems follow the minimality principle, providing only
%a minimal set of abstractions. Minimizing the trusted TCB size
%significantly reduces the attack surface and increases the overall
%security of the system. The reduced TCB size also allows for easier
%security audits. Compared to on-the-metal microkernels, we rely on the
%existing monolithic kernel (\ie Linux, OS X or Windows) to do the
%``heavy lifting'' which means that the various microkernel-related
%overheads (\eg context switch) still exists. \todo{Some readers will
  %complain that we're not reducing the TCB size, since this ``heavy
  %lifting'' code is definitely part of the TCB.  The exposition should
  %be clear that we are referring to the additional TCB needed to
  %implement the VM.}  The benefits of our approach are mainly
%structural; since the VM is implemented entirely in the user space, it
%does not require any special privileges. This makes component/module
%replacement significantly easier.

%By providing a POSIX compliant environment, it will allow execution of
%the existing native application with minimal CPU, memory and I/O
%overhead making it a lightweight alternative to existing hypervisors.
%Since the VM provides resource abstraction decoupling the application
%from the host OS, it allows the same executable to run on top of three
%popular operating systems. This makes it an excellent alternative to
%existing language runtimes (\eg JVM). By targeting a subset of x86, it
%does not impose any restrictions on the target language.  Furthermore,
%by providing direct access to CPU and memory, it allows developers to
%leverage the underlying hardware including extended instruction sets
%(\eg SSE, AVX).

%%Combined with compiler infrastructure such as LLVM, which already
%%contains support for Native Client implemented as a part of
%%PNaCl~\cite{donovan:pnacl10}, it could form a basis for portable and
%%secure execution runtime for new programming languages.

%Based on top of Native Client~\cite{yee:ieee-sp09}, a state of the art
%open-source industrial strength SFI implementation, this solution can
%provide security and performance assurances comparable to existing
%system virtual machines at a fractional \emph{trusted code base} (TCB)
%size which allows for easier verification and maintenance. 

%Providing the external interface will allow embedding
%the VM into existing applications \ldots

%Virtualization became the way to solve the problem of portability by
%abstracting the underlying software and/or hardware upon which
%our applications run.

%The three key abstractions that a microkernel should
%provide~\cite{liedtke:sosp93} are address spaces, threads and
%inter-process communication (IPC).

%Our proposed solution extends Native Client (NaCl), an open-source
%production-quality SFI-based sandbox for the safe execution of
%untrusted, multi-threaded user-level machine code. While primarily aimed
%towards executing native compiled code inside the web browser, the NaCl
%core components, depicted in Figure~\ref{fig:architecture}, can be also
%used as a general-purpose sandbox.

%In this position paper, we argue for a lightweight VM capable of
%executing native code providing the standard POSIX interface. The
%original goal of POSIX standard was to maintain compatibility between
%operating systems, hence the name \emph{Portable Operating System
%Interface}. While it has not fully succeeded for a number of different
%reasons (\eg ABI differences, non-standard extensions), we believe it
%still represents an ideal interface for application execution
%environment. 

%Over the past few years, Java virtual machine (JVM) with its ``write
%once, run everywhere'' capability became a popular platform for
%development of cross-platform applications as well as a popular target
%as a runtime environment for new programing languages. While providing
%secure environment was one of the original JVM goals, the number of
%security vulnerabilities found in older as well as recent versions shows
%otherwise. That is why existing PaaS providers such as Google App Engine
%run Java applications in custom sandboxed environment. Furthermore,
%since JVM was designed originally for Java language, despite recent
%improvements such as the Da Vinci Machine
%Project\footnote{\url{http://openjdk.java.net/projects/mlvm/}}, it still
%imposes number of restrictions on the languages implemented on top (\eg
%type erasure). The situation is similar with other language runtime such
%as CPython or YARV.

%Since our environment targets subset of x86, it does not impose any
%restrictions on target language.  Combined with compiler infrastructure
%such as LLVM, which already contains support for Native Client
%implemented as a part of PNaCl~\cite{donovan:pnacl10}, it could form a
%basis for portable and secure execution runtime for new
%programming languages.

%Furthermore, native code support together with the full POSIX interface
%allows running existing language runtimes on top of our VM
%without the necessity of ad-hoc sandboxing mechanism since the
%environment itself already provides sandboxing capabilities. This has a
%number of advantages.  First, it is possible to use ``vanilla'' language
%runtimes which allows for upgrading to a newer version immediately after
%they are released significantly reducing the risk of zero-day
%vulnerabilities. Second, it allows provisioning of wider range of
%runtimes potentially attracting more customers, even those using less
%popular languages and platforms.  Third, it will allow use of native
%extensions supported by most language runtimes, but not always available
%in PaaS environments for security reasons.

%The NaCl architecture resembles existing micro and hybrid kernels.
%However, compared to on-the-metal microkernels, there are number of
%significant differences. First of all, Native Client sits on top of an
%existing monolithic kernel (\ie Linux, OS X or Windows), so the various
%microkernel-related overheads (\eg context switch) still exist.

%The solution based on Native Client would allow executing native
%POSIX-compliant code at near native speed while providing the same
%security and isolation guarantees. This will allow the use of wide
%range of existing legacy software without the overhead of running (and
%managing) the entire VM.

%Based on top of Native Client~\cite{yee:ieee-sp09}, state of the art
%industrial strength SFI implementation, this solution can provide
%security and performance assurances comparable to existing system
%virtual machines at a fractional \emph{trusted code base} (TCB) size
%which allows for easier verification and maintenance. 

%The NaCl architecture, depicted in Figure~\ref{fig:architecture},
%comprises of many different components resembling different components
%of an operating system. The component which fulfills the role of kernel
%is the \emph{service runtime}.
% largely resembles existing monolithic and micro-kernel design.

%The NaCl architecture resembles existing micro and hybrid kernels.
%However, compared to on-the-metal microkernels, there are number of
%significant differences. First of all, Native Client sits on top of an
%existing monolithic kernel (\ie Linux, OS X or Windows), so the various
%microkernel-related overheads (\eg context switch) still exist.

%\todo{Explain how the NaCl architecture compares with on-the-metal
%  micro-kernel designs, viz: Native Client is on top of an existing
%  monolithic kernel, so the various microkernel-related overheads
%  (context switch, etc) still exist.  The benefits are structural:
%  easier replacement of components/modules, easier security audits.}

%Native Client follows micro-kernel design by implementing many aspects
%of the sandbox in terms of services. All services are implemented using
%SRPC (Simple RPC) which \ldots implemented on top of the IMC
%(Inter-Module Communications). SRPC abstraction is implemented entirely
%in untrusted code. SRPC provides a convenient syntax for declaring
%procedural interfaces between NaCl modules, supporting a few basic types
%(int, float, char) as well as arrays in addition to NaCl descriptors.
%More complex types and pointers are not supported. External data
%representation strategies can easily be layered on top of SRPC.

%\subsection{Service Runtime}

%The service runtime is a native executable which provides resource
%abstractions to isolate NaCl applications from host resources and
%underlying operating system interfaces. Even though it shares a process
%with the contained NaCl module, the service runtime prevents untrusted
%code from inappropriate accesses using a combination of various
%techniques. % which differ for various target architectures. The service
%% runtime trusted code and data are only accessible through a controlled
%% interface.

%The service runtime functionality is only accessible through system
%calls which resemble standard kernel system calls by performing a contex
%switch from untrusted to trusted code, even though they use different
%hardware mechanism~\cite{yee:ieee-sp09,sehr:usenix-sec10}. The service
%runtime Application Binary Interface (ABI) exposed to the untrusted code
%is modeled after POSIX ABI and includes subset of the POSIX thread
%interface as well as common POSIX file I/O interfaces. This makes it
%easy to port existing POSIX application over to Native Client with
%minimal or no changes at all.

%%While Native Client already resembles existing hybrid kernel design
%%providing facilities such as multi-threading and
%%virtual memory management, it lacks two important features, process
%%management and shared memory. \todo{Not sure whether shared memory
%%really that important?}

%\todo{Each service runtime instance has two channels---secure command
%channel used by reverse service to control the execution, in particular
%to load the NaCl executable and intergrated runtime, start and stop the
%module execution; and reverse channel used by service runtime to access
%kernel functionality.}

%\subsection{Reverse Service}

%% While service runtime is the in-process sandbox enforcing fault
%% isolation and providing facilities such basic IPC (SRPC),
%% virtual memory, resource abstraction or the system call interface, it is
%% the \emph{reverse service} which provides inter-process services such as
%% file access, application IPC or process support.

%The \emph{reverse service} represents the embedding interface, providing
%core facilities such as process management, file system access or
%application. IPC % thereby resembling the traditional micro-kernels. 
%The name comes from the fact that unlike other service which act as a
%server, the reverse service starts as a client connecting to a service
%runtime which acts a server and only then is the connection
%\emph{reversed} and reverse service becomes the server. This setup is
%necessary since service runtime is not allowed to establish outside
%connection as one of the security measures.

%Each service runtime instance has a connection to reverse service. The
%\emph{initial} service runtime has a connection to the trusted reverse
%service \ldots while child processes might be using a different instance
%of the reverse service. This allows \ldots injecting a proxy which
%intercepts some (or all) calls to the actual reverse service or provide
%entirely different reverse service implementation running in untrusted
%code.

%\subsection{Integrated Runtime}

%Integrated Runtime (IRT) is an untrusted library providing a stable,
%backward compatible interface to NaCl module. IRT ensures that once
%compiled, NaCl module will run forever even with a newer version of NaCl
%runtime. Application programmers do not typically use IRT directly,
%instead they use one of the provided C libraries (Newlib or Glibc) which
%hide away internal details and provide common interface.

%Facilities provided to untrusted code may be implemented either as NaCl
%systeem calls or as SRPC services. IRT abstracts these differences and
%provides unified interface. The advantage of using services over system
%calls is extensibility and easier development, the disadvantage is
%primarily performance overhead as each service method invocation
%requires several system calls (\ie sequence of \lstinline`sendmsg`
%/\lstinline`recvmsg` invocations).

%\todo{Describe the role of IRT. Describe name and kernel services, their
%role.}
