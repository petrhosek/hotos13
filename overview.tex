\begin{structure}
  \item What is NaCl and how does it work?
  \item NaCl architecture and components (with figure)
  \item Process support issues (\eg no fork on Windows)
  \item Proposed approach with process service
  \item Memory sharing between processes
  \item Memory service design
\end{structure}

Native Client, commonly abbreviated ``NaCl'' when used as an adjective,
is an open-source production-quality SFI-based sandbox for the safe
execution of untrusted, multi-threaded user-level machine
code~\cite{yee:ieee-sp09,sehr:usenix-sec10,ansel:pldi11}. While
primarily aimed towards executing native compiled code inside the web
browser, the NaCl core components can be also used as
general-purpose sandbox.

NaCl architecture, depicted in Figure \ldots largely resemebles
existing monolithic- and micro-kernel design.

\todo{Explain how the NaCl architecture compares with on-the-metal
  micro-kernel designs, viz: Native Client is on top of an existing
  monolithic kernel, so the various microkernel-related overheads
  (context switch, etc) still exist.  The benefits are structural:
  easier replacement of components/modules, easier security audits.}
