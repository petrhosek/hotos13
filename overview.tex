\section{Overview}
\label{sec:overview}

\begin{structure}
  \item What is NaCl and how does it work?
  \item NaCl architecture and components (with figure)
  \item Process support issues (\eg no fork on Windows)
  \item Proposed approach with process service
  \item Memory sharing between processes
  \item Memory service design
\end{structure}

Native Client, commonly abbreviated ``NaCl'' when used as an adjective,
is an open-source production-quality SFI-based sandbox for the safe
execution of untrusted, multi-threaded user-level machine
code~\cite{yee:ieee-sp09,sehr:usenix-sec10,ansel:pldi11}. While
primarily aimed towards executing native compiled code inside the web
browser, the NaCl core components can be also used as a
general-purpose sandbox.

\begin{figure}
\centering
%\includegraphics{architecture}
\caption{Overview of core NaCl components}
\label{fig:architecture}
\end{figure}

The NaCl architecture, depicted in Figure~\ref{fig:architecture},
comprises of many different components resembling different components
of an operating system. The component which fulfills the role of kernel
is the \emph{service runtime}.
% largely resembles existing monolithic and micro-kernel design.

Native Client follows micro-kernel design by implementing many aspects
of the sandbox in terms of services. All services are implemented using
SRPC (Simple RPC), a simple and secure remote method invocation
service. SRPC is implemented entirely in untrusted code on top the IMC
(Inter-Module Communications), Native Client's reliable datagram
service. SRPC provides a convenient syntax for declaring procedural
interfaces between NaCl modules, supporting a few basic types (int,
float, char) as well as arrays in addition to NaCl descriptors.  More
complex types and pointers are not supported.  External data
representation strategies can easily be layered on top of SRPC.

%Native Client follows micro-kernel design by implementing many aspects
%of the sandbox in terms of services. All services are implemented using
%SRPC (Simple RPC) which \ldots implemented on top of the IMC
%(Inter-Module Communications). SRPC abstraction is implemented entirely
%in untrusted code. SRPC provides a convenient syntax for declaring
%procedural interfaces between NaCl modules, supporting a few basic types
%(int, float, char) as well as arrays in addition to NaCl descriptors.
%More complex types and pointers are not supported. External data
%representation strategies can easily be layered on top of SRPC.

\subsection{Service Runtime}

The service runtime is a native executable which provides resource
abstractions to isolate NaCl applications from host resources and
underlying operating system interfaces. Even though it shares a process
with the contained NaCl module, the service runtime prevents untrusted
code from inappropriate accesses using a combination of various
techniques. % which differ for various target architectures. The service
% runtime trusted code and data are only accessible through a controlled
% interface.

The service runtime functionality is only accessible through system
calls which resemble standard kernel system calls by performing a contex
switch from untrusted to trusted code, even though they use different
hardware mechanism~\cite{yee:ieee-sp09,sehr:usenix-sec10}. The service
runtime Application Binary Interface (ABI) exposed to the untrusted code
is modeled after POSIX ABI and includes subset of the POSIX thread
interface as well as common POSIX file I/O interfaces. This makes it
easy to port existing POSIX application over to Native Client with
minimal or no changes at all.

%While Native Client already resembles existing hybrid kernel design
%providing facilities such as multi-threading and
%virtual memory management, it lacks two important features, process
%management and shared memory. \todo{Not sure whether shared memory
%really that important?}

\todo{Explain how the NaCl architecture compares with on-the-metal
  micro-kernel designs, viz: Native Client is on top of an existing
  monolithic kernel, so the various microkernel-related overheads
  (context switch, etc) still exist.  The benefits are structural:
  easier replacement of components/modules, easier security audits.}

\subsection{Reverse Service}

% While service runtime is the in-process sandbox enforcing fault
% isolation and providing facilities such basic IPC (SRPC),
% virtual memory, resource abstraction or the system call interface, it is
% the \emph{reverse service} which provides inter-process services such as
% file access, application IPC or process support.

The \emph{reverse service} represents the embedding interface, providing
core facilities such as process management, file system access or
application. IPC % thereby resembling the traditional micro-kernels. 
The name comes from the fact that unlike other service which act as a
server, the reverse service starts as a client connecting to a service
runtime which acts a server and only then is the connection
\emph{reversed} and reverse service becomes the server. This setup is
necessary since service runtime is not allowed to establish outside
connection as one of the security measures.

Each service runtime instance has a connection to reverse service. The
\emph{initial} service runtime has a connection to the trusted reverse
service \ldots while child processes might be using a different instance
of the reverse service. This allows \ldots injecting a proxy which
intercepts some (or all) calls to the actual reverse service or provide
entirely different reverse service implementation running in untrusted
code.

\subsection{Integrated Run-Time}

\todo{Describe the role of IRT. Describe name and kernel services, their
role.}

\subsection{Process Support}
\label{sub:process_support}

While existing NaCl implementation already provides thread support,
exposed through a subset of the POSIX threads interface, it is missing
process management support. The process support is necessary in cases
when resource sharing, in particular of virtual memory, is not desired.

The process support is important for certain set of applications such
as interactive shells. These shells typically rely on \lstinline`fork` to
spawn new application processes. 

\todo{This introduction is not particularly amazing.}

The NaCl Application Binary Interface (ABI) exposed to untrusted code is
modeled after POSIX ABI. This makes it easy to port existing POSIX
application over to Native Client with minimal or no changes at all.
% Service runtime provides a subset of the POSIX threads interface as
% well as common POSIX file I/O interface.

Therefore, following the same philosophy, our main goal when designing
the process support for NaCl was to provide an equivalent of the POSIX
\lstinline`fork` system call. However, there are two challenges.

First, the process support has to be portable. Native Client executable
can be run on any supported platform without recompilation (as long as
the architecture remains the same). This means that process support can
only rely on a common denominator of the functionality provided by the
supported underlying operating systems (\ie Windows, Linux and OS X).

Different operating systems provide different process creation
primitives. For example, while operating systems from the UNIX family
typically use the \lstinline`fork` semantics, Windows family operating
systems provide \lstinline`spawn` facilities. Even though POSIX
specification provides both \lstinline`fork` and \lstinline`posix_spawn`,
only OS X as fully POSIX conformant provides both primitives.

Second, the process support needs to have a clear semantics. This is an
issue with \lstinline`fork` operation which is supposed to create the
exact copy the existing process. However, when using a \lstinline`fork`
with multi-threaded program, we might end up in an inconsistent state
since only the thread invoking \lstinline`fork` is cloned.

This problem is even worse in case of NaCl service runtime which shares
the process with untrusted code and uses automatic resource management
to increase reliability of resource (de)allocation. However, if we use
the \lstinline`fork` system call to create a copy of the existing process,
the bookkeeping information will become invalid (\eg memory mappings,
descriptors, locks) leading to potentially inconsistent state.

To address these challenges, instead of providing the
\lstinline`fork`-like facility, we have decided to provide set of basic
primitives which can be used to construct the new process in truly
micro-kernel fashion, and rely on untrusted libraries to implement the
\lstinline`fork`-like semantics (\eg or different semantics if needed) on
top of these.

\ldots

Since parent process is responsible for setting up the reverse service
connection, it can subsume the role of reverse service itself and either
proxy some of the root reverse service operations or completely replace
it.

\ldots

Since reverse service is being used for accessing file system \ldots we
can effectively create a sub-sandbox for the child process. one example
computational kernel (seccomp).

\subsection{Memory Sharing}
\label{sub:memory_sharing}

\begin{structure}
\item Memory Service used for fork emulation;
\item Page protection and user-level exceptions used to detect
  writes to shared pages to emulate copy-on-write;
\item Assumes same protection domain, so badly behaving application
  can only harm themselves.
\end{structure}

User-level fork emulation library manages address space.  All memory
are backed by shared memory, though initially not necessarily actually
shared.  On fork, new subprocess is created which have special IRT
that speaks the fork-emulation protocol.  The subprocess receives
shared memory descriptors and location where they are mapped in the
parent process.  Both parent and child processes change page
protection of all pages to read-only to emulate copy-on-write.  On
fault, affected page is copied to new shared memory, reference count
to the old page within the original shared memory object is
decremented, the new shared memory pages mapped in place with writable
mapping, and the affect thread is continued.  We don't want to have a
separate shared memory file for every 64KB page (probably too many
descriptors, filesystem related overhead), so there will be some
interesting optimizations here: if we shared memory object is multiple
pages in size, but a ``hole'' in the middle has zero references, how
do we reclaim the disk / paging space?

Concerns: shared memory is a scarce resource.  /dev/shm is typically a
small fraction of RAM size.  Threads vs fork -- may need to stop all
other threads while pages are being manipulated?  On Unices where mmap
on top of existing mapping is atomic, it may not be necessary; on
Windows where unmap must occur first and therefore overmapping is not
atomic, this is likely to be necessary.
