% rubber: bibtex.crossrefs 10000
\documentclass[twocolumn,10pt]{article}

%
% Standard packages.
%
\usepackage{xspace}
\usepackage{color}
\usepackage{times}
\usepackage{fullpage}
\usepackage{graphicx}
\usepackage[hyphens]{url}
\usepackage{multirow}
\usepackage{flushend}
\usepackage{textcomp}

\usepackage{listings}
\lstset{
  language=C,
  captionpos=b,
  basicstyle=\ttfamily,
  commentstyle=\textit,
  keywordstyle=\bfseries,
  numbers=left,
  numberstyle=\tiny,
  numbersep=5pt,
  escapeinside={/*@}{@*/},
  tabsize=2,
  breaklines=true,
  extendedchars=true,
  columns=fullflexible,
  xleftmargin=16pt,
  showspaces=false,
  showtabs=false,
  showstringspaces=false,
}

%
% Standard margin stuff.  US letter.
%
\setlength{\topmargin}{0in}
\setlength{\oddsidemargin}{0in}
\setlength{\evensidemargin}{\oddsidemargin}
\setlength{\footskip}{0.5in}
\setlength{\textheight}{8.5in}% 11in - 2x1in margins - .5 for page number
\setlength{\textwidth}{6.5in} % 8.5in - 2x1in margins

\newcommand{\textstt}[1]{{\small \texttt{#1}}\xspace}
\newcommand{\textssc}[1]{{\small \textsc{#1}}\xspace}

\newcommand{\todo}[1]{\noindent{\color{red}[{\bf todo:} #1]}}
\newcommand{\sidenote}[1]{{\marginpar{\bf\color{red}[#1]}}}
\newcommand{\gap}[2]{\uline{#1} \noindent{\color{red}[#2]}}

\newenvironment{structure}{\color{blue}
\begin{list}{\labelitemi}{\leftmargin=1em}{%
\setlength{\topsep}{0pt}\setlength{\itemsep}{0pt}\setlength{\parsep}{0pt}}}
{\end{list}}

\newenvironment{structure*}{\color{blue}
\begin{enumerate}\setlength{\topsep}{0pt}\setlength{\itemsep}{0pt}\setlength{\parsep}{0pt}}
{\end{enumerate}}

\newcommand{\ie}{\emph{i.e.},\ }
\newcommand{\eg}{\emph{e.g.},\ }
\newcommand{\etal}{\emph{et al.}\ }
\newcommand{\vs}{\emph{vs.}\ }
\newcommand{\etc}{\emph{etc.}\ }

\begin{document}

\title{\bf Microkernel in User Space}
\author{
Petr Hosek\footnote{The author conducted this work while employed by
Google.}\\\textit{Imperial College London}
\and Bennet Yee\\\textit{Google Inc.}
}
\date{}
\maketitle
\thispagestyle{empty}

\begin{abstract}
We propose a new type of application virtual machine capable of running
native applications. We aim to combine aspects of a microkernel design
with the existing software fault isolation technology to provide the
POSIX compliant execution environment. Such execution environment
presents an alternative to existing hardware-based hypervisors providing
the strong security assurances at fractional trusted code base size
without the necessity of running an entire operating system. The
portable user space implementation allows native applications to be run
on top of three major operating systems with minimal overhead. Well
defined external interface allows the execution environment to be
embedded with existing systems.

%We propose a new type of application virtual machine capable of running
%native applications. We aim to combine aspects of a microkernel design
%with the existing software fault isolation technology to provide the
%POSIX compliant execution environment delivering the security assurances
%comparable to the existing hardware-based hypervisors with the
%portability of language runtimes.

%Such execution environment presents a  alternative to existing
%hardware-based hypervisors providing the strong security assurances at
%fractional trusted code base size without the necessity of running an
%entire operating system. Well defined external interface allows the
%execution environment to be embedded with an existing systems.

%Our proposed approach is portable, allows applications to be run on top
%of three major operating systems, and does not incur significant
%performance overhead.

%In recent years, together with the rising popularity of cloud computing
%we have witnessed a boom of virtualization. Most of the software
%infrastructure which is being used nowadays runs virtualized in some
%form. However, existing virtualization solutions are still too limited
%despite obvious advantages such as portability, security and isolation.
%Virtual machine monitors require the user to run an entire operating
%system, which is suitable for infrastructure provisioning in data centers
%as well as certain cloud applications, but is simply too cumbersome for
%desktop and client applications.  Language runtimes on the other hand
%are typically targeting only a single high-level language (or family) making
%them unsuitable for native applications.
\end{abstract}

\section{Introduction}
\label{sec:intro}

%With the rising popularity of cloud computing, the popularity of
%virtualization is steadily growing as a way to implement
%\emph{multi-tenancy}.

%In recent years, virtualization has become crucialy important for \ldots
%multi-tenancy, cloud computing.

Virtualization, in its many forms, has become the universal solution to
a number of problems including security and portability.  With the
rising popularity of cloud computing, virtualization has also become
crucially important as a way to ensure \emph{multi-tenant security}. The
\emph{X}aaS applications can run either on top of virtual machine
monitors, where the customer is provided with the entire operating
system (\eg Google Compute Engine) or on top of existing language
runtime (\eg Google App Engine).

While the operating system-level virtualization is desirable in cases
when we require control over the entire execution stack, it can be
simply too cumbersome when we need to execute only a single application.
A typical OS installation may require storage space in the order of GB
plus additional CPU and memory overhead for running the kernel and
system services. Furthermore, the guest OS may require frequent
maintenance (\eg installing updates, managing the configuration).
Language runtimes on the other hand, while being more lightweight and
easier to use, typically only target a single high-level language or
language family. 

In this position paper, we argue for an application virtual machine (VM)
capable of running existing POSIX-based applications. The main goal of
our approach is to offer the same security and isolation assurances as
hardware-based hypervisors (\eg Xen), but implemented entirely in the
user space on top of existing monolithic kernel resembling language
runtimes (\eg JVM). Furthermore, by providing the clean embedding
interface, such a VM could be integrated into both new and existing
systems to allow safe native code execution.

%With the rising popularity of cloud computing, virtualization has become
%crucially important as a way to provide \emph{multi-tenancy}. While the
%\emph{Infrastructure as a Service} (IaaS) model assumes provisioning an
%entire operating system using virtual machine monitors (\eg Google
%Compute Engine), the \emph{Platform as a Service} (PaaS) model typically
%relies on a language runtimes (\eg Google App Engine). The~advantage of
%the former is a finer degree of control over the entire application
%stack while the latter allows for better resource utilization. A~modern
%hypervisor uses hardware virtualization technology to enforce isolation
%and resource allocation; in case of language runtimes, ad-hoc sandboxing
%mechanisms are typically being used to provide the same guarantees,
%which results in a limited number of supported platforms (\eg Java,
%Python and Go in case of Google App Engine).

%As cloud providers can use multi-tenancy run large number of
%applications---potentially coming from different customers---under a
%single operating system instance. The existing PaaS platforms such as
%Google App Engine already uses multi-tenancy where several applications
%share the same physical or virtual machine.  However, to ensure the
%security and isolation, the number of supported platforms is limited to
%Java, Python and Go using custom sandboxed runtimes.

%The existing PaaS platforms such as Google App Engine already uses
%multi-tenancy where several applications share the same physical or
%virtual machine.  However, to ensure the security and isolation, the
%number of supported platforms is limited to Java, Python and Go using
%custom sandboxed runtimes.

Such an execution environment could form the basis for a new kind of
POSIX compliant PaaS native platform which would allow for execution of
high-performance applications. Such a platform would be more efficient
than traditional hardware virtualization platforms and some language
based sandboxes while being more flexible than Linux-based systems, and
would be akin to virtual machines.  Furthermore, native code support
enables the use of existing C/C++ libraries and legacy code as well as
extended instruction sets (\eg SSE or AVX) for high-performance
applications. An example of such a service is Google
Exacycle,\footnote{See
\url{http://research.google.com/university/exacycle_program.html} and
\url{http://googleresearch.blogspot.com/2011/04/1-billion-core-hours-of-computational.html}.}
which provides a PaaS platform for large-scale, embarrassingly parallel
batch computations.

%Furthermore, native code support together with full POSIX interface
%allows running existing runtimes on top of our execution environment
%without the necessity of an ad-hoc sandboxing mechanism since the
%environment itself already provides sandboxing capabilities. This has a
%number of advantages.  First, it is possible to use ``vanilla'' language
%runtimes which allows for upgrading to a newer version immediately after
%they are released significantly reducing the risk of zero-day
%vulnerabilities. Second, it allows provisioning of wider range of
%runtimes potentially attracting more customers, even those using less
%popular languages and platforms.  Third, it will allow use of native
%extensions supported by most language runtimes, but not always available in
%PaaS environments for security reasons.

%The solution based on Native Client would allow executing native
%POSIX-compliant code at near native speed while providing the same
%security and isolation guarantees. This will allow the use of wide
%range of existing legacy software without the overhead of running (and
%managing) the entire VM.

%i like NaCl as a posix-like platform-as-a-service a little better,
%though restricting to the posix-y subset sort of works wrt running on a
%real posix-y OS too, i guess.

%more efficient than virtualization and some language based sandboxes.
%more flexible than LBS, akin to VMs
%not just novelty, but TCB minimization

The rest of this paper is organized as follows.
Section~\ref{sec:motivation} sets out the motivation for our proposed
solution while Section~\ref{sec:overview} gives an overview of the
technical approach.  Then, Section~\ref{sec:prototype} presents a
possible prototype implementation. Finally, Section~\ref{sec:related}
discusses some of the related work and Section~\ref{sec:conclusion}
concludes.

%Furthermore, our approach gives greater level control which is so
%important in cloud deployments.

%In recent years, we have witnessed a boom of virtualization. Most of the
%server infrastructure which is now being used runs virtualized.
%However, virtualization still failed to attract desktop market despite
%obvious advantages such as portability, security, reliability and
%isolation. We believe this is because the existing virtualization
%solutions are too heavyweight and cumbersome. They require the user run
%an entire operating system which is suitable for infrastructure
%provisioning in data centers as well as certain cloud applications, but
%is simply too cumbersome for desktop and client applications.

%In this paper, we propose a more lightweight alternative. Instead of
%running the full operating system, our goal is to combine aspects of
%microkernel and monolithic kernel design with software fault isolation
%techniques to provide lightweight solution for virtualization of
%existing applications in the user space. Furthemore, our proposed
%approach allows applications to be run on top of three major operating
%system without recompilation.

\section{Motivation}
\label{sec:motivation}

%The original goal of POSIX standard was to maintain compatibility
%between operating systems, hence the name \emph{Portable Operating
%System Interface}. Unfortunately, it was not successful for number of
%reasons including system call differences (\eg different numbers,
%non-standard extensions), different binary formats (\eg ELF, Mach-O),
%different libraries and services. Windows systems no longer provide
%POSIX subsystem and replacements such as Cygwin are unofficial, often
%relying on internal undocumented API and they are only partially
%POSIX-compliant.  All this makes porting applications across platforms
%challenging.

%Furthermore, Microsoft POSIX subsystem is no longer supported and
%Subsystem for UNIX-based Applications has been also marked as
%deprecated with the income of Windows 8. Replacements such as Cygwin
%are unofficial, often relying on internal undocumented API  and they
%are only partially POSIX-compliant.

%Java virtual machine (JVM) with its ``write once, run everywhere''
%capability became a viable alternative to POSIX for development of
%cross-platform applications. That is why JVM is now being a popular
%target as a runtime environment for new programming languages. While one
%of the goals of JVM was to provide secure environment, a number of
%security vulnerabilities shows otherwise. Furthermore, since JVM was
%designed originally for Java language, despite recent improvements such
%as the Da Vinci Machine
%Project\footnote{\url{http://openjdk.java.net/projects/mlvm/}}, it
%imposes number of restrictions on the languages implemented on top (\eg
%type erasure). The situation is similar with other language runtime such
%as CPython or YARV.

%Since our environment targets subset of x86, it does not impose any
%restrictions on target language.  Combined with compiler infrastructure
%such as LLVM, which already contains support for Native Client
%implemented as a part of PNaCl~\cite{donovan:pnacl10}, it could form a
%basis for portable and secure execution runtime for new
%programming languages.

%Furthermore, native code support together with full POSIX interface
%allows running existing runtimes on top of our execution environment
%without the necessity of ad-hoc sandboxing mechanism since the
%environment itself already provides sandboxing capabilities. This has a
%number of advantages.  First, it is possible to use ``vanilla'' language
%runtimes which allows for upgrading to a newer version immediately after
%they are released significantly reducing the risk of zero-day
%vulnerabilities. Second, it allows provisioning of wider range of
%runtimes potentially attracting more customers, even those using less
%popular languages and platforms.  Third, it will allow use of native
%extensions supported by most language runtimes, but not always available in
%PaaS environments for security reasons.

%%JVM does not support optimalization such as SSE or AVX
%%CPython, YARV have severe restrictions such single thread (GIL)

%A different way to achieve cross-platform portability is application
%virtualization. This technique typically relies on application streaming
%or desktop virtualization.

%The main disadvatange of desktop virtualization, is the requirement to
%run the entire guest operating system. In situation when typical OS
%installation requires units to tens of GB of storage space and units of
%GB of memory, using a VM for individual applications can certainly be
%considered as overkill. Furthermore, guest OS requires frequent
%maintenance (\eg installing updates, managing configuration); in terms
%of TCB size, typical kernel size can be in order of MLOC together with
%the hypervisor which may have over 1 MLOC as well.

%We can also use emulators such as Wine or Cygwin, which aim to emulate
%the operating system and parts of the environment to 

%Embedding interface as a way to obtain descriptors
%Representing everything as descriptors which can be transfered across
%processes \ldots

%Unlike language runtimes, provides direct CPU and memory access which
%allows for micro-optimization (\eg SSE and AVX)

%cross-platform portable, language neutral

%In this position paper, we argue for a lightweight VM capable of
%executing native code providing the standard POSIX interface.  We aim to
%combine \emph{software fault isolation} (SFI)~\cite{wahbe:sosp93} for
%security and isolation properties with microkernel-style service
%oriented architecture to provide process management, file and network
%access. Based on top of Native Client~\cite{yee:ieee-sp09}, a state of
%the art industrial strength SFI implementation, this solution can
%provide security and performance assurances comparable to existing
%system virtual machines at a fractional \emph{trusted code base} (TCB)
%size which allows for easier verification and maintenance.

Hypervisors these days rely on virtualization hardware to deliver
reasonable performance overhead. This makes the hypervisor the most
privileged process effectively replacing the OS~\cite{heiser:hotos11}.
Needles to say, it makes the hypervisor a critical part of the entire
execution stack.  It does not matter whether the hypervisor is over 1
MLOC~\cite{barham:sosp03} or just a few kLOC~\cite{steinberg:eurosys10},
any vulnerability in the hypervisor implementation may potentially
subvert the security of all hosted operating systems. This is a serious
issue in multi-tenant environment where different host operating systems
may run software from different customers.

We aim for a different approach. Instead of virtualization hardware, we
plan to employ software fault isolation~\cite{wahbe:sosp93} to direct
all system interactions through controlled interfaces. This allows
implementing the VM entirely in the userspace as a regular process
without any access to privileged CPU features. Even if the security of
the VM is compromised, the attacker is still limited by the userspace
boundary. To further increase security, it is possible to employ other
sandboxing mechanisms such as seccomp-bpf~\cite{seccomp-bpf:linux}.

We believe it is indeed possible to implement a simple and secure
virtualization environment sharing the approach with microkernel-based
systems. These systems follow the minimality principle, providing only
a minimal set of abstractions. Minimizing the trusted TCB size
significantly reduces the attack surface and increases the overall
security of the system. The reduced TCB size also allows for easier
security audits. Compared to on-the-metal microkernels, we rely on the
existing monolithic kernel (\ie Linux, OS X or Windows) to do the
``heavy lifting'' which means that the various microkernel-related
overheads (\eg context switch) still exists. \todo{Some readers will
  complain that we're not reducing the TCB size, since this ``heavy
  lifting'' code is definitely part of the TCB.  The exposition should
  be clear that we are referring to the additional TCB needed to
  implement the VM.}  The benefits of our approach are mainly
structural; since the VM is implemented entirely in the user space, it
does not require any special privileges. This makes component/module
replacement significantly easier.

Providing a POSIX compliant environment will allow executing the
existing native applications with minimal CPU, memory and I/O overhead
making it a lightweight alternative to existing hypervisors.  Since the
VM provides resource abstraction decoupling the application from the
host OS, it allows the same executable to run on top of three popular
operating systems. This makes it an excellent alternative to existing
language runtimes (\eg JVM). By targeting a subset of x86, it does not
impose any restrictions on the target language.  Furthermore, by
providing direct access to CPU and memory, it allows developers to
leverage the underlying hardware including extended instruction sets
(\eg SSE, AVX).

%Combined with compiler infrastructure such as LLVM, which already
%contains support for Native Client implemented as a part of
%PNaCl~\cite{donovan:pnacl10}, it could form a basis for portable and
%secure execution runtime for new programming languages.

We argue that our proposal could readily be implemented using Native
Client~\cite{yee:ieee-sp09}, a state of the art open-source industrial
strength SFI implementation. \todo{Explain why Native Client is an ideal
solution in high-level terms, \eg part of Google Chrome used by millions
of users, periodically security audited.}

\todo{Talk about use as an embedded VM to allow use of native code (as
well as interpreted languages) in applications. Native Client runtime at
the moment weights \~500kB which makes it perfectly suitable.}

%The features we want to provide are flexibility and maintainability.
%\ldots embedding interface \ldots allows to use \ldots in various setups
%\ie executing native code within a web page, lightweight virtualization
%in cloud provisioning

%Different implementations of embedding interface, one for browser, one
%for server, one for mobile phone, etc.

%Based on top of Native Client~\cite{yee:ieee-sp09}, a state of the art
%open-source industrial strength SFI implementation, this solution can
%provide security and performance assurances comparable to existing
%system virtual machines at a fractional \emph{trusted code base} (TCB)
%size which allows for easier verification and maintenance. 

%Providing the external interface will allow embedding
%the VM into existing applications \ldots

%Virtualization became the way to solve the problem of portability by
%abstracting the underlying software and/or hardware upon which
%our applications run.

%The three key abstractions that a microkernel should
%provide~\cite{liedtke:sosp93} are address spaces, threads and
%inter-process communication (IPC).

%Our proposed solution extends Native Client (NaCl), an open-source
%production-quality SFI-based sandbox for the safe execution of
%untrusted, multi-threaded user-level machine code. While primarily aimed
%towards executing native compiled code inside the web browser, the NaCl
%core components, depicted in Figure~\ref{fig:architecture}, can be also
%used as a general-purpose sandbox.

%In this position paper, we argue for a lightweight VM capable of
%executing native code providing the standard POSIX interface. The
%original goal of POSIX standard was to maintain compatibility between
%operating systems, hence the name \emph{Portable Operating System
%Interface}. While it has not fully succeeded for a number of different
%reasons (\eg ABI differences, non-standard extensions), we believe it
%still represents an ideal interface for application execution
%environment. 

%Over the past few years, Java virtual machine (JVM) with its ``write
%once, run everywhere'' capability became a popular platform for
%development of cross-platform applications as well as a popular target
%as a runtime environment for new programing languages. While providing
%secure environment was one of the original JVM goals, the number of
%security vulnerabilities found in older as well as recent versions shows
%otherwise. That is why existing PaaS providers such as Google App Engine
%run Java applications in custom sandboxed environment. Furthermore,
%since JVM was designed originally for Java language, despite recent
%improvements such as the Da Vinci Machine
%Project\footnote{\url{http://openjdk.java.net/projects/mlvm/}}, it still
%imposes number of restrictions on the languages implemented on top (\eg
%type erasure). The situation is similar with other language runtime such
%as CPython or YARV.

%Since our environment targets subset of x86, it does not impose any
%restrictions on target language.  Combined with compiler infrastructure
%such as LLVM, which already contains support for Native Client
%implemented as a part of PNaCl~\cite{donovan:pnacl10}, it could form a
%basis for portable and secure execution runtime for new
%programming languages.

%Furthermore, native code support together with the full POSIX interface
%allows running existing language runtimes on top of our VM
%without the necessity of ad-hoc sandboxing mechanism since the
%environment itself already provides sandboxing capabilities. This has a
%number of advantages.  First, it is possible to use ``vanilla'' language
%runtimes which allows for upgrading to a newer version immediately after
%they are released significantly reducing the risk of zero-day
%vulnerabilities. Second, it allows provisioning of wider range of
%runtimes potentially attracting more customers, even those using less
%popular languages and platforms.  Third, it will allow use of native
%extensions supported by most language runtimes, but not always available
%in PaaS environments for security reasons.

%The NaCl architecture resembles existing micro and hybrid kernels.
%However, compared to on-the-metal microkernels, there are number of
%significant differences. First of all, Native Client sits on top of an
%existing monolithic kernel (\ie Linux, OS X or Windows), so the various
%microkernel-related overheads (\eg context switch) still exist.

%The solution based on Native Client would allow executing native
%POSIX-compliant code at near native speed while providing the same
%security and isolation guarantees. This will allow the use of wide
%range of existing legacy software without the overhead of running (and
%managing) the entire VM.

%Based on top of Native Client~\cite{yee:ieee-sp09}, state of the art
%industrial strength SFI implementation, this solution can provide
%security and performance assurances comparable to existing system
%virtual machines at a fractional \emph{trusted code base} (TCB) size
%which allows for easier verification and maintenance. 

%The NaCl architecture, depicted in Figure~\ref{fig:architecture},
%comprises of many different components resembling different components
%of an operating system. The component which fulfills the role of kernel
%is the \emph{service runtime}.
% largely resembles existing monolithic and micro-kernel design.

%The NaCl architecture resembles existing micro and hybrid kernels.
%However, compared to on-the-metal microkernels, there are number of
%significant differences. First of all, Native Client sits on top of an
%existing monolithic kernel (\ie Linux, OS X or Windows), so the various
%microkernel-related overheads (\eg context switch) still exist.

%\todo{Explain how the NaCl architecture compares with on-the-metal
%  micro-kernel designs, viz: Native Client is on top of an existing
%  monolithic kernel, so the various microkernel-related overheads
%  (context switch, etc) still exist.  The benefits are structural:
%  easier replacement of components/modules, easier security audits.}

%Native Client follows micro-kernel design by implementing many aspects
%of the sandbox in terms of services. All services are implemented using
%SRPC (Simple RPC) which \ldots implemented on top of the IMC
%(Inter-Module Communications). SRPC abstraction is implemented entirely
%in untrusted code. SRPC provides a convenient syntax for declaring
%procedural interfaces between NaCl modules, supporting a few basic types
%(int, float, char) as well as arrays in addition to NaCl descriptors.
%More complex types and pointers are not supported. External data
%representation strategies can easily be layered on top of SRPC.

%\subsection{Service Runtime}

%The service runtime is a native executable which provides resource
%abstractions to isolate NaCl applications from host resources and
%underlying operating system interfaces. Even though it shares a process
%with the contained NaCl module, the service runtime prevents untrusted
%code from inappropriate accesses using a combination of various
%techniques. % which differ for various target architectures. The service
%% runtime trusted code and data are only accessible through a controlled
%% interface.

%The service runtime functionality is only accessible through system
%calls which resemble standard kernel system calls by performing a contex
%switch from untrusted to trusted code, even though they use different
%hardware mechanism~\cite{yee:ieee-sp09,sehr:usenix-sec10}. The service
%runtime Application Binary Interface (ABI) exposed to the untrusted code
%is modeled after POSIX ABI and includes subset of the POSIX thread
%interface as well as common POSIX file I/O interfaces. This makes it
%easy to port existing POSIX application over to Native Client with
%minimal or no changes at all.

%%While Native Client already resembles existing hybrid kernel design
%%providing facilities such as multi-threading and
%%virtual memory management, it lacks two important features, process
%%management and shared memory. \todo{Not sure whether shared memory
%%really that important?}

%\todo{Each service runtime instance has two channels---secure command
%channel used by reverse service to control the execution, in particular
%to load the NaCl executable and intergrated runtime, start and stop the
%module execution; and reverse channel used by service runtime to access
%kernel functionality.}

%\subsection{Reverse Service}

%% While service runtime is the in-process sandbox enforcing fault
%% isolation and providing facilities such basic IPC (SRPC),
%% virtual memory, resource abstraction or the system call interface, it is
%% the \emph{reverse service} which provides inter-process services such as
%% file access, application IPC or process support.

%The \emph{reverse service} represents the embedding interface, providing
%core facilities such as process management, file system access or
%application. IPC % thereby resembling the traditional micro-kernels. 
%The name comes from the fact that unlike other service which act as a
%server, the reverse service starts as a client connecting to a service
%runtime which acts a server and only then is the connection
%\emph{reversed} and reverse service becomes the server. This setup is
%necessary since service runtime is not allowed to establish outside
%connection as one of the security measures.

%Each service runtime instance has a connection to reverse service. The
%\emph{initial} service runtime has a connection to the trusted reverse
%service \ldots while child processes might be using a different instance
%of the reverse service. This allows \ldots injecting a proxy which
%intercepts some (or all) calls to the actual reverse service or provide
%entirely different reverse service implementation running in untrusted
%code.

%\subsection{Integrated Runtime}

%Integrated Runtime (IRT) is an untrusted library providing a stable,
%backward compatible interface to NaCl module. IRT ensures that once
%compiled, NaCl module will run forever even with a newer version of NaCl
%runtime. Application programmers do not typically use IRT directly,
%instead they use one of the provided C libraries (Newlib or Glibc) which
%hide away internal details and provide common interface.

%Facilities provided to untrusted code may be implemented either as NaCl
%systeem calls or as SRPC services. IRT abstracts these differences and
%provides unified interface. The advantage of using services over system
%calls is extensibility and easier development, the disadvantage is
%primarily performance overhead as each service method invocation
%requires several system calls (\ie sequence of \lstinline`sendmsg`
%/\lstinline`recvmsg` invocations).

%\todo{Describe the role of IRT. Describe name and kernel services, their
%role.}

\section{Native Client as a VM}
\label{sec:overview}

Native Client comprises a sandbox which supports a restricted
subset of x86, x86-64 and ARM ISA defined by a set of constraints, a
modified compilation tool chain generating code that observes these
constrains, and a static validator that verifies that the constrains
have been followed.  The sandbox is implemented as the part of the
\emph{service runtime}, a native executable which provides resource
abstractions to isolate NaCl applications from host resources and
underlying operating system interfaces. Even though it shares a process
with the contained NaCl application, the service runtime prevents
untrusted code from inappropriate accesses using various hardware
mechanisms~\cite{yee:ieee-sp09,sehr:usenix-sec10}. The service runtime
functionality is accessible through system calls which resemble standard
kernel system calls by performing a ``context switch'' from untrusted to
trusted code.

%The Application Binary Interface (ABI) exposed to the untrusted code is
%modeled after POSIX ABI and includes subset of the POSIX thread
%interface as well as common POSIX file I/O interfaces.

%\begin{figure}
%\centering
%%\includegraphics{architecture}
%\caption{Overview of core NaCl components}
%\label{fig:overview}
%\end{figure}

The service runtime architecture resembles a microkernel providing
the three key abstractions as defined by Liedtke~\cite{liedtke:sosp93}:
\begin{inparaenum}[(i)]
\item address spaces, used to ensure that all data accesses are properly
  sandboxed;
\item threads as a mechanism for allocating CPU, available through a subset
  of POSIX thread interface; and
\item IMC, a cross-platform socket implementation, and SRPC, a simple
  remote method invocation implemented on top of IMC, both of
  which can be used as an IPC mechanism.
\end{inparaenum}

The service runtime implementation  should be seen as an attempt at a
minimalist TCB in user space, but not necessarily the smallest possible
one. The service runtime provides a ``wide'' range of system calls and
does not convert all kernel invocations into messages like Mach 3.0 and
other pure microkernels; it only uses messaging in cases when the
communication overhead is negligible compared to the operation itself.
The reason is fully pragmatic. Since it is difficult to implement secure
and fast cross-platform IPC entirely in the user space, the service
runtime instead exposes descriptors, which are mostly used as
capabilities, together with a set of operations implemented as system
calls. When seen as an efficient way to implement IPC on top of existing
monolithic kernel, the service runtime could be considered as a
microkernel; otherwise it should be treated as a hybrid architecture.

%The service runtime resembles a microkernel. According to
%Liedtke~\cite{liedtke:sosp93}, microkernel should provide three key
%abstractions: address spaces, threads and inter-process communication
%(IPC). The service runtime satisfies this definition. It manages the
%process' address space to ensure that all data accesses are properly
%sandboxed.  Threads are available through a subset of POSIX thread
%interface.  Finally, it implements certain services using Simple RPC
%(SRPC), a simple and basic remote method invocation mechanism, which
%could be used as an IPC mechanism.

What is missing to allow Native Client being used as a portable VM for
native applications is the standard POSIX API; the existing Application
Binary Interface (ABI) exposed to the untrusted code contains only a
subset of POSIX ABI including the POSIX thread interface and common
POSIX file I/O interfaces. While this makes it possible to port some of
the existing POSIX based code, it is not enough to enable the use of
existing applications simply by recompiling them using NaCl tool chain.

Rather than implementing the missing interfaces as a part of the service
runtime and thereby unnecessarily increasing the TCB size, we believe
that it is possible to implement the missing POSIX interfaces using the
microkernel approach by implementing most of the functionality in terms of
services (or \emph{servers} in microkernel terms).  These services
should be a part of an \emph{embedding interface}, a set of services the
embedder of Native Client provides to the NaCl runtime. The service
runtime may then mediate access to (some of) these services. The
embedders may also choose to  omit some of these services in situations
when certain interfaces are not available (\eg filesystem or network
access).

The goal of the embedding interface is to provide API sets needed by the
class of applications that embedder want to run. The POSIX API can be
seen as a basic interface necessary to allow porting of the existing
native code. However, additional API such as OpenGL might be needed on a
per-embedding or per-application basis. The goal of the embedding
interface is to allow easy implementation of these additional interfaces
rather then trying to provide a fixed set. This is where microkernel
approach is particularly suitable.

To abstract away the internal details Native Client provides
\emph{Integrated Runtime} (IRT), an untrusted library providing a
stable, backward compatible interface to a NaCl module. IRT ensures that
once compiled, a NaCl module will run forever even with a newer version
of service runtime. Facilities provided to untrusted code may be
implemented either as NaCl system calls or as SRPC services. IRT
abstracts these differences and provides a single unified interface.
Therefore, IRT could be really seen as a part of the embedding
interface. We expect embedders to provide their own version of IRT which
will use the services provided as a part of their embedding interface
implementation. It is also possible to implement certain POSIX
interfaces entirely as a part of IRT (\eg in-memory filesystem).

%The features we want to provide are flexibility and maintainability.
%\ldots embedding interface \ldots allows to use \ldots in various setups
%\ie executing native code within a web page, lightweight virtualization
%in cloud provisioning

%Different implementations of embedding interface, one for browser, one
%for server, one for mobile phone, etc.

%What is missing to provide untrusted code with the POSIX compliant
%interface are services (or \emph{servers} in microkernel terms) for
%process support, file system and network access. These services should
%be a part of \emph{embedding interface}, a set of services the embedder
%of Native Client provides to the NaCl runtime. The service runtime may
%then mediate access to (some of) these services.

%Although Native Client already provides a subset of POSIX interface,
%which allows porting some of the existing POSIX based code, it is still
%missing certain interfaces which makes it difficult to run existing
%applications simply by recompiling them using NaCl tool chain.  What is
%missing to provide untrusted code with the POSIX compliant interface are
%services (or \emph{servers} in microkernel terms) for process support,
%file system and network access. These services should be a part of
%\emph{embedding interface}, a set of services the embedder of Native
%Client provides to the NaCl runtime. The service runtime may then
%mediate access to (some of) these services.

%The Application Binary Interface (ABI) exposed to the untrusted code is
%modeled after POSIX ABI and includes subset of the POSIX thread
%interface as well as common POSIX file I/O interfaces.  However,
%application programmers do not typically use the ABI directly, instead
%they use one of the provided C libraries (\ie Newlib or Glibc) which hide
%away internal details and provide common interface.

%To abstract away the internal details Native Client provides
%\emph{Integrated Runtime} (IRT), an untrusted library providing a
%stable, backward compatible interface to a NaCl module. IRT ensures that
%once compiled, NaCl module will run forever even with a newer version of
%service runtime. Facilities provided to untrusted code may be
%implemented either as NaCl system calls or as SRPC services. IRT
%abstracts these differences and provides unified interface. The
%advantage of using services over system calls is extensibility and
%easier development, the disadvantage is primarily performance overhead
%as each service method invocation requires several system calls (\ie
%sequence of \lstinline`sendmsg`/\lstinline`recvmsg` invocations).

%Native Client combines two sandboxing mechanisms. The inner sandbox
%supports restricted subset of x86, x86-64 and ARM ISA defined by a set
%of constraints, a modified compilation tool chain generating code that
%observes these constrains, and a static validator that verifies that
%the constrains have been followed. To sandbox data accesses, Native
%Client uses different hardware mechanisms depending on target hardware
%platform~\cite{yee:ieee-sp09,sehr:usenix-sec10}. For additional
%security, Native Client optionally uses additional outer sandbox
%implemented as a traditional OS system-call interception (\eg using
%seccomp~\cite{seccomp:linux} or
%seccomp\mbox{-}bpf~\cite{seccomp-bpf:linux}).

%The sandbox is implemented as the part of \emph{service runtime}, a
%native executable which provides resource abstractions to isolate NaCl
%applications from host resources and underlying operating system
%interfaces. Even though it shares a process with the contained NaCl
%module, the sandbox prevents untrusted code from inappropriate accesses.
%The service runtime functionality is only accessible through system
%calls which resemble standard kernel system calls by performing a
%``context switch'' from untrusted to trusted code, even though they use
%a different hardware mechanism. The service runtime implements some
%aspects of the sandbox in terms of services (\eg name service). All
%services are implemented using Simple RPC (SRPC), a simple and basic
%remote method invocation mechanism supporting only basic data types (\ie
%\lstinline`int`, \lstinline`float`, \lstinline`char` in addition to
%arrays and internal descriptors).

%In this position paper, we argue for a lightweight VM capable of
%executing native code providing the standard POSIX interface.  We aim to
%combine \emph{software fault isolation} (SFI)~\cite{wahbe:sosp93} for
%security and isolation properties with microkernel-style service
%oriented architecture to provide process management, file and network
%access. Based on top of Native Client~\cite{yee:ieee-sp09}, a state of
%the art industrial strength SFI implementation, this solution can
%provide security and performance assurances comparable to existing
%system virtual machines at a fractional \emph{trusted code base} (TCB)
%size which allows for easier verification and maintenance.

%Hypervisors these days rely on virtualization hardware to deliver
%reasonable performance overhead. This makes the hypervisor the most
%privileged process effectively replacing the OS~\cite{heiser:hotos11}.
%Needles to say, it makes the hypervisor a critical part of the entire
%execution stack.  It does not matter whether the hypervisor is over 1
%MLOC~\cite{barham:sosp03} or just a few kLOC~\cite{steinberg:eurosys10},
%any vulnerability in the hypervisor implementation may potentially
%subvert the security of all hosted operating systems. This is a serious
%issue in multi-tenant environment where different host operating systems
%may run software from different customers.

%We aim for a different approach. Instead of virtualization hardware, we
%plan to employ software fault isolation~\cite{wahbe:sosp93} to direct
%all system interactions through controlled interfaces. This allows
%implementing the VM entirely in the userspace as a regular process
%without any access to privileged CPU features. Even if the security of
%the VM is compromised, the attacker is still limited by the userspace
%boundary. To further increase security, it is possible to employ other
%sandboxing mechanisms such as seccomp-bpf~\cite{seccomp-bpf:linux}.

%We believe it is indeed possible to implement a simple and secure
%virtualization environment sharing the approach with microkernel-based
%systems. These systems follow the minimality principle, providing only
%a minimal set of abstractions. Minimizing the trusted TCB size
%significantly reduces the attack surface and increases the overall
%security of the system. The reduced TCB size also allows for easier
%security audits. Compared to on-the-metal microkernels, we rely on the
%existing monolithic kernel (\ie Linux, OS X or Windows) to do the
%``heavy lifting'' which means that the various microkernel-related
%overheads (\eg context switch) still exists. \todo{Some readers will
  %complain that we're not reducing the TCB size, since this ``heavy
  %lifting'' code is definitely part of the TCB.  The exposition should
  %be clear that we are referring to the additional TCB needed to
  %implement the VM.}  The benefits of our approach are mainly
%structural; since the VM is implemented entirely in the user space, it
%does not require any special privileges. This makes component/module
%replacement significantly easier.

%By providing a POSIX compliant environment, it will allow execution of
%the existing native application with minimal CPU, memory and I/O
%overhead making it a lightweight alternative to existing hypervisors.
%Since the VM provides resource abstraction decoupling the application
%from the host OS, it allows the same executable to run on top of three
%popular operating systems. This makes it an excellent alternative to
%existing language runtimes (\eg JVM). By targeting a subset of x86, it
%does not impose any restrictions on the target language.  Furthermore,
%by providing direct access to CPU and memory, it allows developers to
%leverage the underlying hardware including extended instruction sets
%(\eg SSE, AVX).

%%Combined with compiler infrastructure such as LLVM, which already
%%contains support for Native Client implemented as a part of
%%PNaCl~\cite{donovan:pnacl10}, it could form a basis for portable and
%%secure execution runtime for new programming languages.

%Based on top of Native Client~\cite{yee:ieee-sp09}, a state of the art
%open-source industrial strength SFI implementation, this solution can
%provide security and performance assurances comparable to existing
%system virtual machines at a fractional \emph{trusted code base} (TCB)
%size which allows for easier verification and maintenance. 

%Providing the external interface will allow embedding
%the VM into existing applications \ldots

%Virtualization became the way to solve the problem of portability by
%abstracting the underlying software and/or hardware upon which
%our applications run.

%The three key abstractions that a microkernel should
%provide~\cite{liedtke:sosp93} are address spaces, threads and
%inter-process communication (IPC).

%Our proposed solution extends Native Client (NaCl), an open-source
%production-quality SFI-based sandbox for the safe execution of
%untrusted, multi-threaded user-level machine code. While primarily aimed
%towards executing native compiled code inside the web browser, the NaCl
%core components, depicted in Figure~\ref{fig:architecture}, can be also
%used as a general-purpose sandbox.

%In this position paper, we argue for a lightweight VM capable of
%executing native code providing the standard POSIX interface. The
%original goal of POSIX standard was to maintain compatibility between
%operating systems, hence the name \emph{Portable Operating System
%Interface}. While it has not fully succeeded for a number of different
%reasons (\eg ABI differences, non-standard extensions), we believe it
%still represents an ideal interface for application execution
%environment. 

%Over the past few years, Java virtual machine (JVM) with its ``write
%once, run everywhere'' capability became a popular platform for
%development of cross-platform applications as well as a popular target
%as a runtime environment for new programing languages. While providing
%secure environment was one of the original JVM goals, the number of
%security vulnerabilities found in older as well as recent versions shows
%otherwise. That is why existing PaaS providers such as Google App Engine
%run Java applications in custom sandboxed environment. Furthermore,
%since JVM was designed originally for Java language, despite recent
%improvements such as the Da Vinci Machine
%Project\footnote{\url{http://openjdk.java.net/projects/mlvm/}}, it still
%imposes number of restrictions on the languages implemented on top (\eg
%type erasure). The situation is similar with other language runtime such
%as CPython or YARV.

%Since our environment targets subset of x86, it does not impose any
%restrictions on target language.  Combined with compiler infrastructure
%such as LLVM, which already contains support for Native Client
%implemented as a part of PNaCl~\cite{donovan:pnacl10}, it could form a
%basis for portable and secure execution runtime for new
%programming languages.

%Furthermore, native code support together with the full POSIX interface
%allows running existing language runtimes on top of our VM
%without the necessity of ad-hoc sandboxing mechanism since the
%environment itself already provides sandboxing capabilities. This has a
%number of advantages.  First, it is possible to use ``vanilla'' language
%runtimes which allows for upgrading to a newer version immediately after
%they are released significantly reducing the risk of zero-day
%vulnerabilities. Second, it allows provisioning of wider range of
%runtimes potentially attracting more customers, even those using less
%popular languages and platforms.  Third, it will allow use of native
%extensions supported by most language runtimes, but not always available
%in PaaS environments for security reasons.

%The NaCl architecture resembles existing micro and hybrid kernels.
%However, compared to on-the-metal microkernels, there are number of
%significant differences. First of all, Native Client sits on top of an
%existing monolithic kernel (\ie Linux, OS X or Windows), so the various
%microkernel-related overheads (\eg context switch) still exist.

%The solution based on Native Client would allow executing native
%POSIX-compliant code at near native speed while providing the same
%security and isolation guarantees. This will allow the use of wide
%range of existing legacy software without the overhead of running (and
%managing) the entire VM.

%Based on top of Native Client~\cite{yee:ieee-sp09}, state of the art
%industrial strength SFI implementation, this solution can provide
%security and performance assurances comparable to existing system
%virtual machines at a fractional \emph{trusted code base} (TCB) size
%which allows for easier verification and maintenance. 

%The NaCl architecture, depicted in Figure~\ref{fig:architecture},
%comprises of many different components resembling different components
%of an operating system. The component which fulfills the role of kernel
%is the \emph{service runtime}.
% largely resembles existing monolithic and micro-kernel design.

%The NaCl architecture resembles existing micro and hybrid kernels.
%However, compared to on-the-metal microkernels, there are number of
%significant differences. First of all, Native Client sits on top of an
%existing monolithic kernel (\ie Linux, OS X or Windows), so the various
%microkernel-related overheads (\eg context switch) still exist.

%\todo{Explain how the NaCl architecture compares with on-the-metal
%  micro-kernel designs, viz: Native Client is on top of an existing
%  monolithic kernel, so the various microkernel-related overheads
%  (context switch, etc) still exist.  The benefits are structural:
%  easier replacement of components/modules, easier security audits.}

%Native Client follows micro-kernel design by implementing many aspects
%of the sandbox in terms of services. All services are implemented using
%SRPC (Simple RPC) which \ldots implemented on top of the IMC
%(Inter-Module Communications). SRPC abstraction is implemented entirely
%in untrusted code. SRPC provides a convenient syntax for declaring
%procedural interfaces between NaCl modules, supporting a few basic types
%(int, float, char) as well as arrays in addition to NaCl descriptors.
%More complex types and pointers are not supported. External data
%representation strategies can easily be layered on top of SRPC.

%\subsection{Service Runtime}

%The service runtime is a native executable which provides resource
%abstractions to isolate NaCl applications from host resources and
%underlying operating system interfaces. Even though it shares a process
%with the contained NaCl module, the service runtime prevents untrusted
%code from inappropriate accesses using a combination of various
%techniques. % which differ for various target architectures. The service
%% runtime trusted code and data are only accessible through a controlled
%% interface.

%The service runtime functionality is only accessible through system
%calls which resemble standard kernel system calls by performing a contex
%switch from untrusted to trusted code, even though they use different
%hardware mechanism~\cite{yee:ieee-sp09,sehr:usenix-sec10}. The service
%runtime Application Binary Interface (ABI) exposed to the untrusted code
%is modeled after POSIX ABI and includes subset of the POSIX thread
%interface as well as common POSIX file I/O interfaces. This makes it
%easy to port existing POSIX application over to Native Client with
%minimal or no changes at all.

%%While Native Client already resembles existing hybrid kernel design
%%providing facilities such as multi-threading and
%%virtual memory management, it lacks two important features, process
%%management and shared memory. \todo{Not sure whether shared memory
%%really that important?}

%\todo{Each service runtime instance has two channels---secure command
%channel used by reverse service to control the execution, in particular
%to load the NaCl executable and intergrated runtime, start and stop the
%module execution; and reverse channel used by service runtime to access
%kernel functionality.}

%\subsection{Reverse Service}

%% While service runtime is the in-process sandbox enforcing fault
%% isolation and providing facilities such basic IPC (SRPC),
%% virtual memory, resource abstraction or the system call interface, it is
%% the \emph{reverse service} which provides inter-process services such as
%% file access, application IPC or process support.

%The \emph{reverse service} represents the embedding interface, providing
%core facilities such as process management, file system access or
%application. IPC % thereby resembling the traditional micro-kernels. 
%The name comes from the fact that unlike other service which act as a
%server, the reverse service starts as a client connecting to a service
%runtime which acts a server and only then is the connection
%\emph{reversed} and reverse service becomes the server. This setup is
%necessary since service runtime is not allowed to establish outside
%connection as one of the security measures.

%Each service runtime instance has a connection to reverse service. The
%\emph{initial} service runtime has a connection to the trusted reverse
%service \ldots while child processes might be using a different instance
%of the reverse service. This allows \ldots injecting a proxy which
%intercepts some (or all) calls to the actual reverse service or provide
%entirely different reverse service implementation running in untrusted
%code.

%\subsection{Integrated Runtime}

%Integrated Runtime (IRT) is an untrusted library providing a stable,
%backward compatible interface to NaCl module. IRT ensures that once
%compiled, NaCl module will run forever even with a newer version of NaCl
%runtime. Application programmers do not typically use IRT directly,
%instead they use one of the provided C libraries (Newlib or Glibc) which
%hide away internal details and provide common interface.

%Facilities provided to untrusted code may be implemented either as NaCl
%systeem calls or as SRPC services. IRT abstracts these differences and
%provides unified interface. The advantage of using services over system
%calls is extensibility and easier development, the disadvantage is
%primarily performance overhead as each service method invocation
%requires several system calls (\ie sequence of \lstinline`sendmsg`
%/\lstinline`recvmsg` invocations).

%\todo{Describe the role of IRT. Describe name and kernel services, their
%role.}

\section{Prototype}
\label{sec:prototype}

\begin{structure}
\item Process support issues (\eg no fork on Windows)
\item Proposed approach with process service
\item Memory sharing between processes
\item Memory service design
\end{structure}

The goal is to provide full POSIX interface. Although Native Client
already provides a subset of this interface, which allows porting some
of the existing POSIX based code, it is still missing certain interfaces
which makes it difficult to run existing applications simply by
recompiling them using NaCl tool chain.  We believe that it is possible
to implement the missing functionality using microkernel design to
provide a fast yet portable and safe alternative to the existing VM
implementations.

\todo{Introduce embedding interface as a realization of microkernel
design and a way to provide extensibility and flexibility.}
%The features we want to provide are flexibility and maintainability.
%\ldots embedding interface \ldots allows to use \ldots in various setups
%\ie executing native code within a web page, lightweight virtualization
%in cloud provisioning

%Different implementations of embedding interface, one for browser, one
%for server, one for mobile phone, etc.

\begin{figure}
\centering
%\includegraphics{architecture}
\caption{Embedding interface architecture}
\label{fig:architecture}
\end{figure}

While service runtime is the in-process sandbox enforcing fault
isolation and providing facilities such as network communication, memory
management, resource abstraction or the system call interface, it is the
\emph{reverse service} which provides inter-process services such as
file access, process support or IPC facilities.

When started, each service runtime instance is identified by the
secure socket address which is used to establish the \emph{secure
command channel}. This channel is used to control the runtime execution,
in particular to load the NaCl executable and IRT into memory, to
establish the reverse channel, to start and stop the process execution.
The \emph{reverse channel} can be used by the service runtime to access
the embedding interface.

The process holding the secure command channel can be seen as an
effective process parent. While it is indeed possible to have multiple
command channels, there has to be at least one channel at any given
point.  When the last command channel is disconnected, the service
runtime immediately terminates the execution.

%Each service runtime instance has a connection to reverse service. The
%\emph{initial} service runtime has a connection to the trusted reverse
%service \ldots while child processes might be using a different instance
%of the reverse service. This allows \ldots injecting a proxy which
%intercepts some (or all) calls to the actual reverse service or provide
%entirely different reverse service implementation running in untrusted
%code.

%The \emph{reverse service} represents the embedding interface, providing
%core facilities such as process management, file system access or
%application. IPC % thereby resembling the traditional micro-kernels.
%The name comes from the fact that unlike other service which act as a
%server, the reverse service starts as a client connecting to a service
%runtime which acts a server and only then is the connection
%\emph{reversed} and reverse service becomes the server. This setup is
%necessary since service runtime is not allowed to establish outside
%connection as one of the security measures.

\subsection{Process Support}
\label{sub:process_support}

While existing NaCl implementation already supports threads, it misses
process support, important for certain set of applications such as
interactive shells.  \todo{This introduction is not particularly
amazing.}

First and the most important requirement imposed on process support is
portability. Native Client executable must run on any supported platform
without recompilation (as long as the architecture remains the same).
This means that process support can only rely on a common denominator of
the functionality provided by the supported underlying operating systems
(\ie Windows, Linux and OS X). The POSIX specification defines two
interfaces for creating new processes, \lstinline`fork` and
\lstinline`posix_spawn`. While the former creates a copy of an existing
process, the latter creates a new process from the specified image.
However, different operating systems implement different subset of these
primitives. Linux only provides \lstinline`fork` (or \lstinline`clone`
as a non-standard extension).  Windows \lstinline`CreateProcess` is
equivalent to POSIX \lstinline`posix_spawn`. Only OS X as a fully POSIX
conformant OS provides both primitives.

Second, the process support must have a clear semantics. This is an
issue with \lstinline`fork` which is supposed to create the exact copy
the existing process. However, when using a \lstinline`fork` with a
multi-threaded program, we might end up in an inconsistent state since
only the thread invoking \lstinline`fork` is cloned.  This problem is
magnified in case of NaCl service runtime which shares the process with
untrusted code and uses automatic resource management to increase the
reliability of resource (de)allocation. If we use the \lstinline`fork`
system call to create a copy of the existing process, the bookkeeping
information will become invalid (\eg memory mappings, descriptors,
locks) leading to potentially inconsistent state if
\lstinline`pthread_atfork` is not used everywhere that it is needed.

%Therefore, following the same philosophy, our main goal when designing
%the process support for NaCl was to provide an equivalent of the POSIX
%\lstinline`fork` system call. However, there are two challenges.

To address these challenges, instead of implementing \lstinline`fork`
like system call, embedding interface will provide process service  with
a set of methods which may be used to spawn a new instance of
service runtime returning the secure socket address. Since these methods
are accessible to applications, we can rely on untrusted libraries (\eg
IRT) to implement whatever semantics is needed on top these primitives.

Once the new service runtime has been created, the invoking process can
decide whether it will establish the secure command channel and become
the parent, or whether it will pass the socket address to another
process.  This allows creating arbitrary process hierarchy.  While the
default IRT implementation will resembles the standard POSIX behavior,
embedders may provide custom IRT implementation which will use a
different strategy.

The parent process is responsible for loading the executable image and
IRT over the command channel and subsequently starting the execution.
This allows implementing the \lstinline`posix_spawn` like function
entirely in the untrusted code. To implement \lstinline`fork`, we need a
way to share memory between processes provided by \emph{memory service},
further discussed in the next section.

Since the parent process is responsible for setting up the reverse
service connection, it can subsume the role of embedding interface itself
and either proxy some of the services or completely replace them. This
allows creating specialized sub-sandboxes, \eg for executing computation
kernels with restricted priviliges emulating seccomp-like behavior
entirely in the untrusted code.

\subsection{Memory Service}
\label{sub:memory_service}

\begin{structure}
\item Memory Service used for fork emulation;
\item Page protection and user-level exceptions used to detect
  writes to shared pages to emulate copy-on-write;
\item Assumes same protection domain, so badly behaving application
  can only harm themselves.
\end{structure}

To allow implementing \lstinline`fork` emulation including copy-on-write
semantics, embedding interface needs to provide the \emph{memory
service}. When used, all process memory will be backed by a shared
memory, thought initially not necessarily shared.

When a new subprocess is created, it receives shared memory descriptors
together with the location in the parent process' address space. Both
parent and child processes have to change page protection of all memory
pages to read-only to emulate copy-on-write. On fault, affected page is
copied to a new shared memory which belongs to the subprocess, the
reference count to the old page within the original shared memory object
is decremented, the new shared memory pages are mapped in place of the
original pages with write permissions, and the affected thread is
continued.

The memory service will be responsible for managing the shared memory,
in particular (de)allocating the shared memory and automatic reference
management. This is necessary since shared memory is potentially a
scarce resource. There is also a space for potentially interesting
optimizations. First, we do not want a separate shared memory file for
every page (\eg limit on number of file descriptors, filesystem-related
overhead), but if the shared memory object is multiple pages in size and
a ``hole'' in the middle has zero references, we need to find a way to
reclaim the space. Second, we have to make sure that the memory
management is safe with respect to threads. This is not an issue on
systems where altering the existing mapping can be done atomically, in
particular on Linux and OS X (\ie mmap on top of existing mapping).
However, on Windows where memory has to be unmapped first and therefore
over-mapping is not atomic, we have to stop all threads while memory
pages are being manipulated.

%User-level fork emulation library manages address space.  All memory
%are backed by shared memory, though initially not necessarily actually
%shared.  On fork, new subprocess is created which have special IRT
%that speaks the fork-emulation protocol.  The subprocess receives
%shared memory descriptors and location where they are mapped in the
%parent process.  Both parent and child processes change page
%protection of all pages to read-only to emulate copy-on-write.  On
%fault, affected page is copied to new shared memory, reference count
%to the old page within the original shared memory object is
%decremented, the new shared memory pages mapped in place with writable
%mapping, and the affect thread is continued.  We don't want to have a
%separate shared memory file for every 64KB page (probably too many
%descriptors, filesystem related overhead), so there will be some
%interesting optimizations here: if we shared memory object is multiple
%pages in size, but a ``hole'' in the middle has zero references, how
%do we reclaim the disk / paging space?

%Concerns: shared memory is a scarce resource.  /dev/shm is typically a
%small fraction of RAM size.  Threads vs fork -- may need to stop all
%other threads while pages are being manipulated?  On Unices where mmap
%on top of existing mapping is atomic, it may not be necessary; on
%Windows where unmap must occur first and therefore overmapping is not
%atomic, this is likely to be necessary.

\subsection{File Service}
\label{sub:file_service}

\subsection{Network Service}
\label{sub:network_service}

\section{Related Work}
\label{sec:related}

There are number of alternative SFI implementations with different level
of maturity, thought none of them being of the production quality.
\todo{List them as a set of references.}

\todo{Talk about ZeroVM sharing similar set of goals, thought trying to
achieve them in a different way.}

Dune~\cite{belay:osdi12} uses virtualization hardware, in particular
Intel VT-x, to create secure subdomain within a process similarly to
Native Client which relies entirely on software protection. Both
systems run on top of existing kernel and could be considered type-2
hypervisors. The consequence of using virtualization hardware is
twofold; Dune has slightly better performance overhead, but has very
limited portability; the current implementation only supports Linux
running on 64-bit x86 Intel CPUs. Unlike Native Client, which runs
entirely in user-space, Dune is implemented as a Linux kernel module and
runs in a priviledged mode. Therefore, its use has larger security and
administration implications (see Section~\ref{sec:overview}). Dune uses
standard \lstinline`fork` semantics where newly spawned processes are
\emph{not} implicitly sandboxed.

User Mode Linux (UML)~\cite{uml:linux06} enables Linux kernel to be run as
an user-space application to create multiple virtual systems. While the
overall approach is similar to Native Client with each guest process being
mapped to a single host process and guest kernel being responsible for
providing resource abstractions, there are significant differences.
First of all, UML is restricted to Linux with hardware support limited
to 64-bit Intel x86 CPUs, the support for other hardware platforms is
unclear. Second, the performance overhead is larger than in case of
Native Client. Last but not least, the size of Linux kernel code-base is
several order of magnitudes larger than that of Native Client which
makes any security audits merely impossible and also significantly
increases attack space. Moreover, UML does not employ any additional
security mechanisms and relies entirely on user-space isolation. The
advantage of UML over Native Client is full compatibility with existing
applications.

\cite{heiser:hotos11}
\cite{tanenbaum:osdi08}
\cite{engler:sosp95}
\cite{yee:ieee-sp09}
\cite{sehr:usenix-sec10}
\cite{ansel:pldi11}
\cite{donovan:pnacl10}

\section{Conclusion}
\label{sec:conclusion}

In this position paper, we have proposed a new type of application VM
combining SFI with microkernel-style service oriented archicture
providing the POSIX interfaces. The implementation based on Native
Client would allow executing native POSIX compliant code at near native
speed while providing the same security and isolation guarantees as
existing hypervisors. This will allow the use of wide range of existing
legacy software without the overhead of running (and managing) the
entire operating system. Furthermore, our approach gives greater control
than traditional language runtimes which can be useful in a number of
scenarios, in particular as a new type of PaaS native platform. Finally,
by providing a clean embedding interface, the VM could be easily
integrated into other applications providing the native code execution
capability.


{
  \bibliographystyle{abbrv}
  \bibliography{macros,references,crossreferences}
}

\end{document}
