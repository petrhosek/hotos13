\section{The Embedding Interface}
\label{sec:prototype}

%Although Native Client already provides a subset of POSIX interface,
%which allows porting some of the existing POSIX based code, it is still
%missing certain interfaces which makes it difficult to run existing
%applications simply by recompiling them using NaCl tool chain.

%Rather than implementing the missing interfaces as a part of service
%runtime unnecessarily increasing the TCB size, we believe that it is
%possible to implement the missing POSIX interfaces using the microkernel
%design by implementing the majority of functionality in the untrusted
%code to provide a fast yet portable and secure alternative to the
%existing VM implementations.

%While service runtime is the in-process sandbox enforcing fault
%isolation and providing facilities such as network communication, memory
%management, resource abstraction or the system call interface, it is the
%\emph{reverse service} which provides inter-process services such as
%file access, process support or IPC facilities.

%Following the Liedtke's~\cite{liedtke:sosp93} minimality principle, the
%service runtime should provide only the core functionality:
%\begin{itemize}
%\item managing the address space, in particular the memory protection to
	%ensure that all data accesses are properly sandboxed; 
%\item threads as an abstraction for allocating CPU;
%\item IPC mechanism necessary to invoke remote services, in Native
	%Client realized through internal descriptors.
%\end{itemize}

%POSIX doesn't include OpenGL. or db access. so POSIX is sort of the
%basic level for porting code easily, but additional APIs such as Pepper
%(though I would prefer to view it as OpenGL implemented using Pepper)
%are needed on a per-embedding or per-application basis.

%if an embedder wants to expose oracle or a mysql interface to untrusted
%code, for example, it should be able to. (the embedder needs to ensure
%the interfaces do proper validation, policy enforcement, etc, of
%course.)

%I think the pt of view is what are the API sets needed by the class of
%applications that embedders want to run. the subset of POSIX that we
%have now is pretty good (so far) for chrome, plus they need OpenGL (or
%Pepper). but a fuller subset of POSIX will improve things, and the
%non-POSIX APIs are application/embedder/domain specific.

In this section, we describe a design for the embedding interface, in
particular services to allow support for process creation and control,
file and directory operations.  All services are implemented using SRPC.
This mechanism only supports basic data types (\ie \lstinline`int`,
\lstinline`float`, \lstinline`char`) in addition to arrays and internal
descriptors. These are being used (mostly) as capabilities. For
efficiency, the operations on descriptors are implemented as NaCl system
calls.

%While the operations on descriptors are implemented as NaCl system calls
%for efficiency, the majority of interfaces provided to untrusted should
%be implemented in terms of IPC as SRPC services. IRT allows us to
%abstract these differences and provides unified interface. The
%advantage of using services over system calls is extensibility and
%easier development, the disadvantage is primarily performance overhead
%as each service method invocation requires several system calls (\ie
%sequence of \lstinline`sendmsg`/\lstinline`recvmsg` invocations).

%The features we want to provide are flexibility and maintainability.
%\ldots embedding interface \ldots allows to use \ldots in various setups
%\ie executing native code within a web page, lightweight virtualization
%in cloud provisioning

%Different implementations of embedding interface, one for browser, one
%for server, one for mobile phone, etc.

\begin{figure}
\centering
%\includegraphics{architecture}
\caption{Embedding interface architecture}
\label{fig:architecture}
\end{figure}

The embedding interface runs as a process separate from the service
runtime instance with each service running in its own thread (or
process), which improves the robustness and security of the system. When
started, service runtime is identified by the socket address, which is
used to establish the \emph{command channel}. This channel is used to
control the runtime execution, in particular to load the NaCl executable
and IRT into memory, to establish the reverse channel and to start and
stop the process execution.  The \emph{reverse channel} is used by the
service runtime to access the embedding interface. The process holding
the command channel can be seen as an effective process parent. While it
is indeed possible to have multiple command channels, there has to be at
least one channel at any given point.  When the last command channel is
disconnected, the service runtime immediately terminates the execution.

%Each service runtime instance has a connection to reverse service. The
%\emph{initial} service runtime has a connection to the trusted reverse
%service \ldots while child processes might be using a different instance
%of the reverse service. This allows \ldots injecting a proxy which
%intercepts some (or all) calls to the actual reverse service or provide
%entirely different reverse service implementation running in untrusted
%code.

%The \emph{reverse service} represents the embedding interface, providing
%core facilities such as process management, file system access or
%application. IPC % thereby resembling the traditional micro-kernels.
%The name comes from the fact that unlike other service which act as a
%server, the reverse service starts as a client connecting to a service
%runtime which acts a server and only then is the connection
%\emph{reversed} and reverse service becomes the server. This setup is
%necessary since service runtime is not allowed to establish outside
%connection as one of the security measures.

\subsection{Process Support}
\label{sub:process_support}

The support for process creation and control is a part of the POSIX.1,
Core Services standard and is used by a large number of applications.
The most important requirement imposed on process support is
portability. Since the NaCl executable must run on any supported
platform without recompilation, the process support can only rely on a
common denominator of the functionality provided by the supported
underlying operating systems (\ie Windows, Linux and OS X). The POSIX
specification defines two interfaces for creating new processes,
\lstinline`fork` and \lstinline`posix_spawn`. While the former creates a
copy of an existing process, the latter creates a new process from the
specified image.  However, different operating systems implement
different subsets of these primitives. Linux only provides
\lstinline`fork` (or \lstinline`clone` as a non-standard extension).
Windows \lstinline`CreateProcess` is equivalent to POSIX
\lstinline`posix_spawn`. Only OS X provides both primitives.

The process support must have a clear semantics, which may be an issue
with \lstinline`fork` which is often assumed to create an exact copy of
an existing process. However, when using a \lstinline`fork` with a
multi-threaded program, we might end up in an inconsistent state since
only the thread invoking \lstinline`fork` is cloned.  This problem is
magnified in the case of the service runtime, which shares the process
with the untrusted code and uses automatic resource management to
increase the reliability of resource (de)allocation. If we use the
\lstinline`fork` system call to create a copy of the existing process,
the bookkeeping information will become invalid (\eg memory mappings,
descriptors, locks) leading to a potentially inconsistent states.
%if \lstinline`pthread_atfork` is not used everywhere that it is needed.

%Therefore, following the same philosophy, our main goal when designing
%the process support for NaCl was to provide an equivalent of the POSIX
%\lstinline`fork` system call. However, there are two challenges.

To implement the process support, the embedding interface must provide a
\emph{process service} with a set of methods which may be used to spawn
a new instance of service runtime returning the socket address. Since
these methods are accessible to applications, we can rely on untrusted
libraries (\ie IRT) to implement whatever semantics is needed on top of
these primitives. Once the new service runtime has been created, the
invoking process can decide whether it will establish the command
channel and become the parent, or whether it will pass the socket
address to another process.  This allows creating arbitrary process
hierarchy.  While the standard IRT implementation is expected to emulate
the standard POSIX behavior, embedders may provide a custom IRT
implementations which will use a different strategy.

The parent process is responsible for loading the executable image and
IRT over the command channel and subsequently starting the execution.
This allows implementation of the \lstinline`posix_spawn` function
entirely in the untrusted code. To implement \lstinline`fork`, we need a
way to share memory between processes, further discussed in the next
section.  Since the parent process is responsible for setting up the
reverse channel connection, it can either pass the socket up to the
embedded interface, which will likely be the default behavior, or it can
subsume the role of embedding interface itself. This allows the parent
to either proxy some of the services or completely replace them. Such
strategy could be used to further restrict the child process privileges,
\eg for executing computation kernels with restricted filesystem access
emulating chroot-like behavior entirely in the untrusted code.

\subsection{Backing Store}
\label{sub:backing_store}

<<<<<<< HEAD
=======
\todo{We may decide to implement this with regular files, for example,
which can be \lstinline`MAP_SHARED` by the untrusted library. Talk about
DSM~\cite{keleher:phd95}}

>>>>>>> 4b81dd8650f20fca421a7db160c651c81efd8b06
When the process spawns a new subprocess using the \lstinline`fork`
operation, the child's address space should be an exact copy of its
parent's.  The inefficient way to implement such behavior would be to
copy the content of parent's address space. The efficient way,
typically used by OS kernels, is to use the copy-on-write mechanism.
To allow implementation of the \lstinline`fork`-like emulation using
the copy-on-write semantics, the embedding interface has to provide a
\emph{backing store} interface for allocating and managing shareable
memory objects. Memory objects are either shared memory or files that
can be memory mapped into the NaCl process. When used, all the process
memory will be backed by shareable memory objects, though memory is
not initially shared.

When a new subprocess is created, it receives the memory object
descriptors together with the location in the parent's process address
space. Both parent and child processes change page protection of all
memory pages to read-only to emulate copy-on-write, which is a
responsibility of the IRT. On fault, the affected page will be copied
to a new shared memory which belongs to the subprocess, the reference
count to the old page within the original shared memory object is
decremented, the new shared memory pages are mapped in place of the
original pages with write permissions, and the affected thread is
continued.

The backing store interface is used by untrusted code to manage the
shared memory, including deallocating the memory objects when they are
no longer in use.  Since processes created with \lstinline`fork`
execute in the same protection domain, misbehavior with respect to the
\lstinline`fork`-emulation protocol is permissible as long as the
effect is merely to cause memory corruption of NaCl subprocesses in
the same domain or to cause resource leaks.

%User-level fork emulation library manages address space.  All memory
%are backed by shared memory, though initially not necessarily actually
%shared.  On fork, new subprocess is created which have special IRT
%that speaks the fork-emulation protocol.  The subprocess receives
%shared memory descriptors and location where they are mapped in the
%parent process.  Both parent and child processes change page
%protection of all pages to read-only to emulate copy-on-write.  On
%fault, affected page is copied to new shared memory, reference count
%to the old page within the original shared memory object is
%decremented, the new shared memory pages mapped in place with writable
%mapping, and the affect thread is continued.  We don't want to have a
%separate shared memory file for every 64KB page (probably too many
%descriptors, filesystem related overhead), so there will be some
%interesting optimizations here: if we shared memory object is multiple
%pages in size, but a ``hole'' in the middle has zero references, how
%do we reclaim the disk / paging space?

%Concerns: shared memory is a scarce resource.  /dev/shm is typically a
%small fraction of RAM size.  Threads vs fork -- may need to stop all
%other threads while pages are being manipulated?  On Unices where mmap
%on top of existing mapping is atomic, it may not be necessary; on
%Windows where unmap must occur first and therefore overmapping is not
%atomic, this is likely to be necessary.

\subsection{File Access}
\label{sub:file_access}

File access is highly application dependent. The embedders may choose to
provide an access to a host OS filesystem (or its subset).  For cloud
deployments, they may use an online storage shared by all VM instances.
On the other hand, for certain type of applications such as computation
kernels which do not need a persistent storage, it is possible to
provide in-memory filesystem implementation.

The embedding interface implementation may provide a \emph{file service}
as a way to expose host's (or other machine's) filesystem through
internal descriptors. Embedders may also implement custom descriptor
type including all operations resembling
FUSE~\cite{fuse}, which provides a similar functionality in Linux kernel.

The file system access may be also implemented entirely in the untrusted
code on top of IPC (\ie SRPC) as in traditional microkernel systems.
This approach is already being used by
NaCl Mounts,\footnote{See \url{http://code.google.com/p/naclports/source/browse/trunk/src/libraries/nacl-mounts/}}
a pluggable NaCl file system implemented on top of PPAPI, which provides
a number of back-end implementations including memory, WebFS or Google
App Engine. The only disadvantage of this approach is slightly lower
performance associated with the IPC overhead. However, in case of remote
filesystems, the overhead is going to be dominated by network latency
which makes this approach a reasonable alternative.

\subsection{Network Access}
\label{sub:network_access}

Network access is in many aspect similar to file access. Same as file
access, it is also highly application dependent. For applications which
do not need a direct access to the network, we can pass in an already
bound socket as one of the NaCl application arguments. Otherwise, the
embedding interface implementation has to provide a \emph{network
service} which will mediate the access to the host's network interface.
Same as in the case of the file access, the network access can be also
implemented either through internal descriptors or using IPC. The
example of the IPC-based implementation are NaCl
Sockets,\footnote{See \url{http://mainroach.blogspot.co.uk/2012/09/nacl-udp-sockets.html}}
implementation of UDP sockets on top of PPAPI.

\todo{Possibly mention the lack of system calls for asynchronous I/O
(\ie select or poll).}
