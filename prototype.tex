\section{Prototype}
\label{sec:prototype}

Native Client combines two sandboxing mechanisms. The inner sandbox
supports restricted subset of x86, x86-64 and ARM ISA defined by a set
of constraints, a modified compilation tool chain generating code that
observes these constrains, and a static validator that verifies that
the constrains have been followed. To sandbox data accesses, Native
Client uses different hardware mechanisms depending on target hardware
platform~\cite{yee:ieee-sp09,sehr:usenix-sec10}. For additional
security, Native Client optionally uses additional outer sandbox
implemented as a traditional OS system-call interception (\eg using
seccomp~\cite{seccomp:linux} or
seccomp\mbox{-}bpf~\cite{seccomp-bpf:linux}).

\begin{figure}
\centering
%\includegraphics{architecture}
\caption{Overview of core NaCl components}
\label{fig:overview}
\end{figure}

The sandbox is implemented as the part of \emph{service runtime}, a
native executable which provides resource abstractions to isolate NaCl
applications from host resources and underlying operating system
interfaces. Even though it shares a process with the contained NaCl
module, the sandbox prevents untrusted code from inappropriate accesses.
The service runtime functionality is only accessible through system
calls which resemble standard kernel system calls by performing a
``context switch'' from untrusted to trusted code, even though they use
a different hardware mechanism. The service runtime implements some
aspects of the sandbox in terms of services (\eg name service). All
services are implemented using Simple RPC (SRPC), a simple and basic
remote method invocation mechanism supporting only basic data types (\ie
\lstinline`int`, \lstinline`float`, \lstinline`char` in addition to
arrays and internal descriptors).

Although Native Client already provides a subset of POSIX interface,
which allows porting some of the existing POSIX based code, it is still
missing certain interfaces which makes it difficult to run existing
applications simply by recompiling them using NaCl tool chain.

Rather than implementing the missing interfaces as a part of service
runtime unnecessarily increasing the TCB size, we believe that it is
possible to implement the missing POSIX interfaces using the microkernel
design by implementing the majority of functionality in the untrusted
code to provide a fast yet portable and secure alternative to the
existing VM implementations.

While service runtime is the in-process sandbox enforcing fault
isolation and providing facilities such as network communication, memory
management, resource abstraction or the system call interface, it is the
\emph{reverse service} which provides inter-process services such as
file access, process support or IPC facilities.

Following the Liedtke's~\cite{liedtke:sosp93} minimality principle, the
service runtime should provide only the core functionality:
\begin{itemize}
\item managing the address space, in particular the memory protection to
	ensure that all data accesses are properly sandboxed; 
\item threads as an abstraction for allocating CPU;
\item IPC mechanism necessary to invoke remote services, in Native
	Client realized through internal descriptors.
\end{itemize}

While the operations on descriptors are implemented as NaCl system calls
for efficiency, the majority of interfaces provided to untrusted should
be implemented in terms of IPC as SRPC services. IRT allows us to
abstract these differences and provides unified interface. The
advantage of using services over system calls is extensibility and
easier development, the disadvantage is primarily performance overhead
as each service method invocation requires several system calls (\ie
sequence of \lstinline`sendmsg`/\lstinline`recvmsg` invocations).

\todo{Needs work.  IRT used above, defined below.}

To abstract away the internal details Native Client provides
\emph{Integrated Runtime} (IRT), an untrusted library providing a
stable, backward compatible interface to a NaCl module. IRT ensures that
once compiled, NaCl module will run forever even with a newer version of
service runtime. Facilities provided to untrusted code may be
implemented either as NaCl system calls or as SRPC services. IRT
abstracts these differences and provides unified interface. The
advantage of using services over system calls is extensibility and
easier development, the disadvantage is primarily performance overhead
as each service method invocation requires several system calls (\ie
sequence of \lstinline`sendmsg`/\lstinline`recvmsg` invocations).

The Application Binary Interface (ABI) exposed to the untrusted code is
modeled after POSIX ABI and includes subset of the POSIX thread
interface as well as common POSIX file I/O interfaces.  However,
application programmers do not typically use the ABI directly, instead
they use one of the provided C libraries (\ie Newlib or Glibc) which hide
away internal details and provide common interface.

The services together with the IRT form an \emph{embedding interface}.
\todo{Perhaps clarify that embedding interface is what services the
  embedder of Native Client provides to the NaCl runtime, and that the
  service runtime may mediate access to (some of) these services.}

\todo{Introduce embedding interface as a realization of microkernel
design and a way to provide extensibility and flexibility.}
%The features we want to provide are flexibility and maintainability.
%\ldots embedding interface \ldots allows to use \ldots in various setups
%\ie executing native code within a web page, lightweight virtualization
%in cloud provisioning

%Different implementations of embedding interface, one for browser, one
%for server, one for mobile phone, etc.

\begin{figure}
\centering
%\includegraphics{architecture}
\caption{Embedding interface architecture}
\label{fig:architecture}
\end{figure}

When started, each service runtime instance is identified by the
secure socket address which is used to establish the \emph{secure
command channel}. This channel is used to control the runtime execution,
in particular to load the NaCl executable and IRT into memory, to
establish the reverse channel, to start and stop the process execution.
The \emph{reverse channel} can be used by the service runtime to access
the embedding interface.

The process holding the secure command channel can be seen as an
effective process parent. While it is indeed possible to have multiple
command channels, there has to be at least one channel at any given
point.  When the last command channel is disconnected, the service
runtime immediately terminates the execution.


Memory service is again largely implemented in the untrusted code
resembling the microkernel architecture.

IRT is a part of the embedding interface

%Each service runtime instance has a connection to reverse service. The
%\emph{initial} service runtime has a connection to the trusted reverse
%service \ldots while child processes might be using a different instance
%of the reverse service. This allows \ldots injecting a proxy which
%intercepts some (or all) calls to the actual reverse service or provide
%entirely different reverse service implementation running in untrusted
%code.

%The \emph{reverse service} represents the embedding interface, providing
%core facilities such as process management, file system access or
%application. IPC % thereby resembling the traditional micro-kernels.
%The name comes from the fact that unlike other service which act as a
%server, the reverse service starts as a client connecting to a service
%runtime which acts a server and only then is the connection
%\emph{reversed} and reverse service becomes the server. This setup is
%necessary since service runtime is not allowed to establish outside
%connection as one of the security measures.

\subsection{Process Support}
\label{sub:process_support}

While existing NaCl implementation already supports threads, it misses
process support, important for certain set of applications such as
interactive shells.  \todo{This introduction is not particularly
amazing.}

First and the most important requirement imposed on process support is
portability. Native Client executable must run on any supported platform
without recompilation (as long as the architecture remains the same).
This means that process support can only rely on a common denominator of
the functionality provided by the supported underlying operating systems
(\ie Windows, Linux and OS X). The POSIX specification defines two
interfaces for creating new processes, \lstinline`fork` and
\lstinline`posix_spawn`. While the former creates a copy of an existing
process, the latter creates a new process from the specified image.
However, different operating systems implement different subset of these
primitives. Linux only provides \lstinline`fork` (or \lstinline`clone`
as a non-standard extension).  Windows \lstinline`CreateProcess` is
equivalent to POSIX \lstinline`posix_spawn`. Only OS X as a fully POSIX
conformant OS provides both primitives.

Second, the process support must have a clear semantics. This is an
issue with \lstinline`fork` which is supposed to create the exact copy
the existing process. However, when using a \lstinline`fork` with a
multi-threaded program, we might end up in an inconsistent state since
only the thread invoking \lstinline`fork` is cloned.  This problem is
magnified in case of NaCl service runtime which shares the process with
untrusted code and uses automatic resource management to increase the
reliability of resource (de)allocation. If we use the \lstinline`fork`
system call to create a copy of the existing process, the bookkeeping
information will become invalid (\eg memory mappings, descriptors,
locks) leading to potentially inconsistent state if
\lstinline`pthread_atfork` is not used everywhere that it is needed.

%Therefore, following the same philosophy, our main goal when designing
%the process support for NaCl was to provide an equivalent of the POSIX
%\lstinline`fork` system call. However, there are two challenges.

To address these challenges, instead of implementing \lstinline`fork`
like system call, embedding interface will provide process service  with
a set of methods which may be used to spawn a new instance of
service runtime returning the secure socket address. Since these methods
are accessible to applications, we can rely on untrusted libraries (\eg
IRT) to implement whatever semantics is needed on top these primitives.

Once the new service runtime has been created, the invoking process can
decide whether it will establish the secure command channel and become
the parent, or whether it will pass the socket address to another
process.  This allows creating arbitrary process hierarchy.  While the
default IRT implementation will resembles the standard POSIX behavior,
embedders may provide custom IRT implementation which will use a
different strategy.

The parent process is responsible for loading the executable image and
IRT over the command channel and subsequently starting the execution.
This allows implementing the \lstinline`posix_spawn` like function
entirely in the untrusted code. To implement \lstinline`fork`, we need a
way to share memory between processes provided by \emph{memory service},
further discussed in the next section.

Since the parent process is responsible for setting up the reverse
service connection, it can subsume the role of embedding interface itself
and either proxy some of the services or completely replace them. This
allows creating specialized sub-sandboxes, \eg for executing computation
kernels with restricted priviliges emulating seccomp-like behavior
entirely in the untrusted code.

\subsection{Memory Service}
\label{sub:memory_service}

\todo{Replace terminology with ``Backing Store''?  We may decide to
  implement this with regular files, for example, which can be
  \lstinline`MAP_SHARED` by the untrusted library.}

To allow implementing \lstinline`fork` emulation including copy-on-write
semantics, embedding interface needs to provide the \emph{memory
service}. When used, all process memory will be backed by a shared
memory, thought initially not necessarily shared.

When a new subprocess is created, it receives shared memory descriptors
together with the location in the parent process' address space. Both
parent and child processes have to change page protection of all memory
pages to read-only to emulate copy-on-write. On fault, affected page is
copied to a new shared memory which belongs to the subprocess, the
reference count to the old page within the original shared memory object
is decremented, the new shared memory pages are mapped in place of the
original pages with write permissions, and the affected thread is
continued.

The memory service will be responsible for managing the shared memory,
in particular (de)allocating the shared memory and automatic reference
management. This is necessary since shared memory is potentially a
scarce resource. There is also a space for potentially interesting
optimizations. First, we do not want a separate shared memory file for
every page (\eg limit on number of file descriptors, filesystem-related
overhead), but if the shared memory object is multiple pages in size and
a ``hole'' in the middle has zero references, we need to find a way to
reclaim the space. Second, we have to make sure that the memory
management is safe with respect to threads. This is not an issue on
systems where altering the existing mapping can be done atomically, in
particular on Linux and OS X (\ie mmap on top of existing mapping).
However, on Windows where memory has to be unmapped first and therefore
over-mapping is not atomic, we have to stop all threads while memory
pages are being manipulated.

%User-level fork emulation library manages address space.  All memory
%are backed by shared memory, though initially not necessarily actually
%shared.  On fork, new subprocess is created which have special IRT
%that speaks the fork-emulation protocol.  The subprocess receives
%shared memory descriptors and location where they are mapped in the
%parent process.  Both parent and child processes change page
%protection of all pages to read-only to emulate copy-on-write.  On
%fault, affected page is copied to new shared memory, reference count
%to the old page within the original shared memory object is
%decremented, the new shared memory pages mapped in place with writable
%mapping, and the affect thread is continued.  We don't want to have a
%separate shared memory file for every 64KB page (probably too many
%descriptors, filesystem related overhead), so there will be some
%interesting optimizations here: if we shared memory object is multiple
%pages in size, but a ``hole'' in the middle has zero references, how
%do we reclaim the disk / paging space?

%Concerns: shared memory is a scarce resource.  /dev/shm is typically a
%small fraction of RAM size.  Threads vs fork -- may need to stop all
%other threads while pages are being manipulated?  On Unices where mmap
%on top of existing mapping is atomic, it may not be necessary; on
%Windows where unmap must occur first and therefore overmapping is not
%atomic, this is likely to be necessary.

\subsection{File Service}
\label{sub:file_service}

\subsection{Network Service}
\label{sub:network_service}
