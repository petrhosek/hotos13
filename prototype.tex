\section{Prototype}
\label{sec:prototype}

\begin{structure}
\item Process support issues (\eg no fork on Windows)
\item Proposed approach with process service
\item Memory sharing between processes
\item Memory service design
\end{structure}

The goal is to provide full POSIX interface. While Native Client already
provides a subset of POSIX API including thread interface and common
file I/O interfaces, it still misses important parts such as process
support, user management or non-blocking I/O. We believe that it is
possible to implement the missing functionality using microkernel design
to provide fast yet portable and safe alternative to the existing VM
implementations.

The solution based on Native Client would allow executing native
POSIX-compliant code at near native speed while providing the same
security and isolation guarantees. This will allow the use of wide
range of existing legacy software without the overhead of running (and
managing) the entire VM.

\subsection{Process Support}
\label{sub:process_support}

While existing NaCl implementation already provides thread support,
exposed through a subset of the POSIX threads interface, it misses
process management support. The process support is necessary in cases
when resource sharing, in particular of virtual memory, is not desired.

The process support is important for certain set of applications such
as interactive shells. These shells typically rely on \lstinline`fork` to
spawn new application processes. 

\todo{This introduction is not particularly amazing.}

The NaCl Application Binary Interface (ABI) exposed to untrusted code is
modeled after POSIX ABI. This makes it easy to port existing POSIX
application over to Native Client with minimal or no changes at all.
% Service runtime provides a subset of the POSIX threads interface as
% well as common POSIX file I/O interface.

Therefore, following the same philosophy, our main goal when designing
the process support for NaCl was to provide an equivalent of the POSIX
\lstinline`fork` system call. However, there are two challenges.

First, the process support has to be portable. Native Client executable
can be run on any supported platform without recompilation (as long as
the architecture remains the same). This means that process support can
only rely on a common denominator of the functionality provided by the
supported underlying operating systems (\ie Windows, Linux and OS X).

Different operating systems provide different process creation
primitives. For example, while operating systems from the UNIX family
typically use the \lstinline`fork` semantics, Windows family operating
systems provide \lstinline`spawn` facilities. Even though POSIX
specification provides both \lstinline`fork` and \lstinline`posix_spawn`,
only OS X as fully POSIX conformant provides both primitives.

Second, the process support needs to have a clear semantics. This is an
issue with \lstinline`fork` operation which is supposed to create the
exact copy the existing process. However, when using a \lstinline`fork`
with multi-threaded program, we might end up in an inconsistent state
since only the thread invoking \lstinline`fork` is cloned.

This problem is even worse in case of NaCl service runtime which shares
the process with untrusted code and uses automatic resource management
to increase reliability of resource (de)allocation. However, if we use
the \lstinline`fork` system call to create a copy of the existing process,
the bookkeeping information will become invalid (\eg memory mappings,
descriptors, locks) leading to potentially inconsistent state if
\lstinline`pthread_atfork` is not used everywhere that it is needed.

To address these challenges, instead of providing the
\lstinline`fork`-like facility, we have decided to provide set of basic
primitives which can be used to construct the new process in truly
micro-kernel fashion, and rely on untrusted libraries to implement the
\lstinline`fork`-like semantics (\eg or different semantics if needed) on
top of these.

\ldots

Since parent process is responsible for setting up the reverse service
connection, it can subsume the role of reverse service itself and either
proxy some of the root reverse service operations or completely replace
it.

\ldots

Since reverse service is being used for accessing file system \ldots we
can effectively create a sub-sandbox for the child process. one example
computational kernel (seccomp).

\subsection{Memory Sharing}
\label{sub:memory_sharing}

\begin{structure}
\item Memory Service used for fork emulation;
\item Page protection and user-level exceptions used to detect
  writes to shared pages to emulate copy-on-write;
\item Assumes same protection domain, so badly behaving application
  can only harm themselves.
\end{structure}

User-level fork emulation library manages address space.  All memory
are backed by shared memory, though initially not necessarily actually
shared.  On fork, new subprocess is created which have special IRT
that speaks the fork-emulation protocol.  The subprocess receives
shared memory descriptors and location where they are mapped in the
parent process.  Both parent and child processes change page
protection of all pages to read-only to emulate copy-on-write.  On
fault, affected page is copied to new shared memory, reference count
to the old page within the original shared memory object is
decremented, the new shared memory pages mapped in place with writable
mapping, and the affect thread is continued.  We don't want to have a
separate shared memory file for every 64KB page (probably too many
descriptors, filesystem related overhead), so there will be some
interesting optimizations here: if we shared memory object is multiple
pages in size, but a ``hole'' in the middle has zero references, how
do we reclaim the disk / paging space?

Concerns: shared memory is a scarce resource.  /dev/shm is typically a
small fraction of RAM size.  Threads vs fork -- may need to stop all
other threads while pages are being manipulated?  On Unices where mmap
on top of existing mapping is atomic, it may not be necessary; on
Windows where unmap must occur first and therefore overmapping is not
atomic, this is likely to be necessary.
