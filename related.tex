\section{Related Work}
\label{sec:related}

The closest in spirit to our proposal is ZeroVM~\cite{zerovm}. The
project aims to provide a lightweight virtualization platform and same
as in our case it builds on top of Native Client.  However, there are
number of significant differences. ZeroVM does not aim to provide full
POSIX, rather its deterministic subset. In terms of the implementation,
ZeroVM is not an extension rather a fork of Native Client, which
replaces or reimplements a large part of the existing runtime. One of
the consequences is the lack of portability as ZeroVM only supports
Linux x86-64 systems at the moment.

There are number of alternative SFI implementations, all providing
similar set of features, though none of them being of the production
quality~\cite{small:coots97,mccamant:usenix-sec06}.  The most relevant
in this context is Vx32~\cite{ford:usenix-atc08}, an SFI implementation,
which shares number of features with Native Client. The goal however is
not to provide entire runtime environment rather than a library which
could be used to sandbox (a part of) the process and confine the guest
code to a custom system call API. 

%This mechanism has been used to
%implement number of applications~\cite{ford:fast05}.

Similarly, there are also many environments based on a VM architecture
that provide excellent security, but are either too
heavyweight~\cite{adl-tabatabai:pldi96,bugnion:tocs97,waldspurger:osdi02,barham:sosp03,ford:fast05}
or support only a single high-level
language~\cite{lindholm:java99,richter:clr10}. Dune~\cite{belay:osdi12}
takes a different spin using virtualization hardware to create secure
subdomain within a process similarly to Native Client, which relies on
software protection only. The consequence of using virtualization
hardware is twofold; Dune has slightly better performance overhead, but
has very limited portability.  The current implementation only supports
Linux running on 64-bit x86 Intel CPUs. Unlike Native Client, which runs
entirely in user-space, Dune is implemented as a Linux kernel module and
runs in a priviledged mode which has a number of security implications
as discussed in Section~\ref{sec:motivation}.

User Mode Linux (UML)~\cite{dike:uml06} enables Linux kernel to be run as
an user-space application to create multiple virtual systems. The
overall approach is similar to Native Client with each guest process being
mapped to a single host process and guest kernel being responsible for
providing resource abstractions. However, Linux TCB size is several
orders of magnitude larger than that of Native Client which makes any
security audits merely impossible. UML is also naturally restricted to
Linux only. A more lightweight alternative is LXC~\cite{lxc}, which aims
to extend chroot capabilities by implementing complete virtual systems,
which include resource management and isolation mechanisms.
Unfortunately, LXC does not yet provide the same security of other
virtualization technologies, including Native Client, as there is no
inherent protection against exploits that allow a user to become the
root.\footnote{See \url{http://wiki.gentoo.org/wiki/LXC}} Furthermore,
this solution is, same as UML, restricted to Linux only. 

%First of all, UML is restricted to Linux with hardware support limited
%to 64-bit Intel x86 CPUs, the support for other hardware platforms is
%unclear. Second, the performance overhead is larger than in case of
%Native Client. Last but not least, the size of Linux kernel code-base is
%several order of magnitudes larger than that of Native Client which
%makes any security audits merely impossible and also significantly
%increases attack space. Moreover, UML does not employ any additional
%security mechanisms and relies entirely on user-space isolation. The
%advantage of UML over Native Client is full compatibility with existing
%applications.

%\cite{heiser:hotos11}
%\cite{tanenbaum:osdi08}
%\cite{engler:sosp95}
%\cite{yee:ieee-sp09}
%\cite{sehr:usenix-sec10}
%\cite{ansel:pldi11}
%\cite{donovan:pnacl10}
