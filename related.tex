\section{Related Work}
\label{sec:related}

There are number of techniques for SFI with different maturity.

Dune~\cite{belay:osdi12} uses virtualization hardware, in particular
Intel VT-x, to create secure subdomain within a process similarly to
Native Client which relies entirely on software protection. Both
systems run on top of existing kernel and could be considered type-2
hypervisors. The consequence of using virtualization hardware is
twofold; Dune has slightly better performance overhead, but has very
limited portability; the current implementation only supports Linux
running on 64-bit x86 Intel CPUs. Unlike Native Client, which runs
entirely in user-space, Dune is implemented as a Linux kernel module and
runs in a priviledged mode. Therefore, its use has larger security and
administration implications (see Section~\ref{sec:overview}). Dune uses
standard \lstinline`fork` semantics where newly spawned processes are
\emph{not} implicitly sandboxed.

User Mode Linux (UML)~\cite{uml:linux} enables Linux kernel to be run as
an user-space application to create multiple virtual systems. While the
overall approach is similar to Native Client with each guest process being
mapped to a single host process and guest kernel being responsible for
providing resource abstractions, there are significant differences.
First of all, UML is restricted to Linux with hardware support limited
to 64-bit Intel x86 CPUs, the support for other hardware platforms is
unclear. Second, the performance overhead is larger than in case of
Native Client. Last but not least, the size of Linux kernel code-base is
several order of magnitudes larger than that of Native Client which
makes any security audits merely impossible and also significantly
increases attack space. Moreover, UML does not employ any additional
security mechanisms and relies entirely on user-space isolation. The
advantage of UML over Native Client is full compatibility with existing
applications.

\cite{tanenbaum:osdi01}
\cite{engler:sosp95}
\cite{yee:ieee-sp09}
\cite{sehr:usenix-sec10}
\cite{ansel:pldi11}
\cite{donovan:pnacl10}
