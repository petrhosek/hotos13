\section{Related Work}
\label{sec:related}

There are number of alternative SFI implementations with different level
of maturity, thought none of them being of a production
quality~\cite{small:coots97,mccamant:usenix-sec06,ford:usenix-atc08}.
Similarly, there are also many environments based on a VM architecture
that provide excellent security, but are either too
heavyweight~\cite{adl-tabatabai:pldi96,bugnion:tocs97,waldspurger:osdi02,barham:sosp03,ford:fast05}
or support only a single high-level
language~\cite{lindholm:java99,richter:clr10}.

The closest in spirit to our proposal is ZeroVM~\cite{zerovm}. The
project aims to provide a lightweight virtualization platform and same
as in our case it builds on top of Native Client.  However, there are
number of significant differences. Rather than an extension of Native
Client, ZeroVM is a fork reimplementing majority of the existing
runtime; it also relies on external libraries, such as \mbox{ZeroMQ},
which has not been security audited. ZeroVM is also not portable only
supporting Linux x86-64 systems. While being marketed as an embeddable
solution, it does not provide a well defined embedding interface.
Finally, ZeroVM does not aim to provide full POSIX, rather its
deterministic subset.

Vx32~\cite{ford:usenix-atc08} is an alternative SFI implementation,
which shares number of features with Native Client. The goal however is
not to provide entire runtime environment rather than a library which
could be used to sandbox (part of) the process.

Dune~\cite{belay:osdi12} uses virtualization hardware, in particular
Intel VT-x, to create secure subdomain within a process similarly to
Native Client which relies entirely on software protection. Both systems
run on top of existing kernel and could be considered type-2
hypervisors. The consequence of using virtualization hardware is
twofold; Dune has slightly better performance overhead, but has very
limited portability. The current implementation only supports Linux
running on 64-bit x86 Intel CPUs. Unlike Native Client, which runs
entirely in user-space, Dune is implemented as a Linux kernel module and
runs in a priviledged mode which has a number of security implications
(see Section~\ref{sec:motivation}). Dune uses standard \lstinline`fork`
semantics where newly spawned processes are \emph{not} implicitly
sandboxed.

\todo{Talk about lxc: Linux Containers.}

User Mode Linux (UML)~\cite{dike:uml06} enables Linux kernel to be run as
an user-space application to create multiple virtual systems. While the
overall approach is similar to Native Client with each guest process being
mapped to a single host process and guest kernel being responsible for
providing resource abstractions, there are significant differences.
First of all, UML is restricted to Linux with hardware support limited
to 64-bit Intel x86 CPUs, the support for other hardware platforms is
unclear. Second, the performance overhead is larger than in case of
Native Client. Last but not least, the size of Linux kernel code-base is
several order of magnitudes larger than that of Native Client which
makes any security audits merely impossible and also significantly
increases attack space. Moreover, UML does not employ any additional
security mechanisms and relies entirely on user-space isolation. The
advantage of UML over Native Client is full compatibility with existing
applications.

\cite{heiser:hotos11}
\cite{tanenbaum:osdi08}
\cite{engler:sosp95}
\cite{yee:ieee-sp09}
\cite{sehr:usenix-sec10}
\cite{ansel:pldi11}
\cite{donovan:pnacl10}
